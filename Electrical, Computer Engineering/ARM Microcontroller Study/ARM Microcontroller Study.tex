\documentclass[10pt, letterpaper]{article}
\usepackage[letterpaper, portrait, margin=1in]{geometry}   %For page Setup
\usepackage[utf8]{inputenc}
\usepackage{amssymb, amsmath}               %For Equations and Formulas
\usepackage{comment}                        %For Commenting
\usepackage{hyperref}                       %For Hyperlinks
\usepackage{listings}                       %For Coding Examples
\usepackage[table]{xcolor}                  %For Coloring Tables
\usepackage{xcolor}                         %For Color Associated with Coding Examples
\usepackage{multicol}                       %For Making Multiple Columns
\usepackage{multirow}                       %Allows for multiple cells in one row in a table
\usepackage{graphicx, epstopdf}                       %Converts eps files to pdf
\epstopdfsetup{update}
%\usepackage{circuitikz}                     %For Adding Circuits (General)
\usepackage[siunitx]{circuitikz}            %For Adding Circuits with Labels

\title{Notes: Introduction to ARM Cortex $^{\text{TM}}$-M Microcontrollers}
\author{K}
\date{November 1, 2020}

\usepackage{natbib}
\usepackage{graphicx}

\hypersetup{                                %Setup for Hyperlink Colors
    colorlinks=true,
    linkcolor=blue,                         %For Hyperlinked Text
    filecolor=magenta,                      %For Text that Hyperlinks to other Files
    urlcolor=cyan,                          %For Hyperlinked Printed URLs
}



\begin{document}

\begin{comment}
\begin{titlepage}
    %\titlepage
    \maketitle
\end{titlepage}
\end{comment}

\maketitle

\tableofcontents{}
\pagebreak

\section{Introduction to Computer Electronics}
\subsection{Review of Electronics}

\begin{gather*}
  V=\text{Voltage (Volts)}, I=\text{Current (Amperes)}, R=\text{Resistance (Ohms/$\Omega$)}\\
  V=IR, ~I=\frac{V}{R}, ~R=\frac{V}{I}
\end{gather*}
\begin{itemize}
  \item Voltage
  \begin{itemize}
    \item potential to cause current to flow, measured between two places
    \item has polarity
  \end{itemize}

  \item Current
  \begin{itemize}
    \item has direction
  \end{itemize}

  \item If the electron flow has stopped resistance is infinite, no electrons flow
  \item If electrons flow freely, resistance is not zero, but some finite amount
  \item As resistance varies so does current
  \item potential is defined as the voltage difference between two places

  \item current/Flow has direction
  \begin{itemize}
    \item Low resistance, High current
    \item High resistance, Low current
    \item Example: Temperature Movement
    \begin{itemize}
      \item $\text{Flow} = \frac{T_1-T_2}{\text{Resistance}}$
      \item $T$ = Temperature
    \end{itemize}
  \end{itemize}

  \item R-Value
  \begin{itemize}
    \item used in insulation put in walls and ceiling of a house
    \item given in units per square area, e.g. $m^2\cdot ^{\circ} C/w$
    \item amount of heat flow across a wall:
    \begin{itemize}
      \item $\text{Flow} = \frac{\text{Area}\cdot (T_1-T_2)}{\text{R-Value}}$
      \item $T$ = Temperature
    \end{itemize}
  \end{itemize}

  \item Power
  \begin{itemize}
    \item $P$ in watts
    \item does not have power or direction
    \begin{gather*}
      \text{$P$=Power(watts), $V$=Voltage(Volts), $I$=Current(Amperes)}\\
      P=VI, ~P=\frac{V^2}{R}, ~P=I^2 \cdot R
    \end{gather*}
  \end{itemize}

  \item Energy
  \begin{itemize}
    \item $E$ in joules
    \item stored in a battery
    \item has neither polarity or direction
    \begin{gather*}
      \text{$E$ = Energy(Joules), $V$ = Voltage(Voltage), $I$ = Current(Amperes), $t$ =time(seconds)}\\
      E=VIt, ~E=Pt
    \end{gather*}
  \end{itemize}

  \item Switch
  \begin{itemize}
    \item used to modify the behavior of a circuit
    \item ON
    \begin{itemize}
      \item closed, resistance is 0, current flows
      \item resistance of a switch is less than 0.1$\Omega$, assume 0 in most cases
    \end{itemize}
    \item OFF
    \begin{itemize}
      \item open, resistance is $\infty$, no current will flow
      \item resistance if greater than 100M$\Omega$, close to $\infty$ therefore assume $\infty$
    \end{itemize}
  \end{itemize}

  \item Rules for solving voltages and currents in a circuit compromised with batteries, switches, and resistors
  \begin{itemize}
    \item \textbf{Current always flows in a loop}
    \begin{itemize}
      \item When there is no loop, no current can flow
    \end{itemize}
    \item \textbf{Kirchoff's Voltage Law (KVL)}
    \begin{itemize}
      \item The sum of the voltages around the loop is zero
    \end{itemize}
    \item \textbf{Kirchoff's Current Law (KCL)}
    \begin{itemize}
      \item The sum of the currents into a node equal the sum of the currents leaving a node
    \end{itemize}
    \item \textbf{Observation:} If at all possible, draw the circuit so curruent flows down across the resistors and switches. As a secondary rule have currents go left to right across resistors and switches.

    \item \textbf{Series Resistance}
    \begin{itemize}
      \item If resistor $R1$ is in series with resistor $R2$, this combination behaves like one resistor with a value equal to $R1 + R2$
      \item $V$ equals $V1 + V2$
      \item By KCL currents through the two resistors are the same
    \end{itemize}

    \item \textbf{Voltage Divider Rule}
      \begin{itemize}
        \item \begin{align*}
          V2 &= I \cdot R2\\
          &= (V/R) \cdot R2\\
          &= V*R2/(R1 + R2)
        \end{align*}
        \item The following are equivalent:\\
          \begin{circuitikz}  \draw
            (0,0) to[R, R=$R1$, V_=$V1$] (0,2)
              to[R, R=$R1$, V_=$V1$] (0,4)
            (4,0) to[R, R=$R1+R2$, V_=$V1+V2$] (4,4)
            ;
          \end{circuitikz}
      \end{itemize}
  \end{itemize}
\end{itemize}

\begin{circuitikz}[scale=2]  \draw
  (0,0) to[battery] (0,4)
    to[ammeter, i_=2<\milli\ampere>] (4,4)
    to[C=3<\farad>] (4,0) -- (3.5,0)
    to[lamp,*-*] (0.5,0) -- (0,0)
  (0.5,0) -- (0.5,-2)
    to[voltmeter, l=3<\kilo\volt>, color=red] (3.5,-2) -- (3.5,0)
  ;
\end{circuitikz}

\end{document}
