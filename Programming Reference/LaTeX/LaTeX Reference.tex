\documentclass[10pt, letterpaper]{article}
\usepackage[utf8]{inputenc}
\usepackage{amssymb, amsmath}               %For Equations and Formulas
\usepackage{comment}                        %For Commenting
\usepackage{hyperref}                       %For Hyperlinks
\usepackage{listings}                       %For Coding Examples
\usepackage[table]{xcolor}                  %For Coloring Tables
\usepackage{xcolor}                         %For Color Associated with Coding Examples
\usepackage{multicol}                       %For Making Multiple Columns
\usepackage{multirow}                       %Allows for multiple cells in one row in a table
\usepackage{graphicx, epstopdf}                       %Converts eps files to pdf
\epstopdfsetup{update}

\title{LaTeX Notes}
\author{K}
\date{March 6, 2020}

\usepackage{natbib}
\usepackage{graphicx}
\usepackage{blindtext}    %For adding lorem Ipsum

\hypersetup{                                %Setup for Hyperlink Colors
    colorlinks=true,
    linkcolor=blue,                         %For Hyperlinked Text
    filecolor=magenta,                      %For Text that Hyperlinks to other Files
    urlcolor=cyan,                          %For Hyperlinked Printed URLs
}
\definecolor{codegreen}{rgb}{0,0.6,0}       %Define Colors For Coding Examples
\definecolor{codegray}{rgb}{0.5,0.5,0.5}
\definecolor{codepurple}{rgb}{0.58,0,0.82}
\definecolor{backcolor}{rgb}{0.75, 1.00, 1.00}
\definecolor{keyword}{rgb}{0.00, 0.50, 1.00}

\lstdefinestyle{basicstyle}{                  %Setup Style for Listings
    backgroundcolor=\color{white},      %Set Background Color
    %commentstyle=\color{codegreen},        %Set Comment Style
    keywordstyle=\color{keyword},           %Set Keyword Style
    %numberstyle=\tiny\color{codegray},     %Set Number Style
    %stringstyle=\color{codepurple},        %Set String Style
    basicstyle=\ttfamily\footnotesize,
    breakatwhitespace=false,
    breaklines=true,
    captionpos=b,
    keepspaces=true,
    numbers=left,                           %Set Line Numbers
    %numbersep=5pt,                         %Set Line Number Separation
    showspaces=false,
    showstringspaces=false,
    showtabs=false,
    tabsize=2,
    frame=double                              %set Frame
}

\lstdefinestyle{blankstyle}{                  %Setup Style for Listings
    backgroundcolor=\color{white},      %Set Background Color
    %commentstyle=\color{codegreen},        %Set Comment Style
    keywordstyle=\color{keyword},           %Set Keyword Style
    %numberstyle=\tiny\color{codegray},     %Set Number Style
    %stringstyle=\color{codepurple},        %Set String Style
    basicstyle=\ttfamily\footnotesize,
    breakatwhitespace=false,
    breaklines=true,
    captionpos=b,
    keepspaces=true,
    numbers=none,                           %Set Line Numbers
    %numbersep=5pt,                         %Set Line Number Separation
    showspaces=false,
    showstringspaces=false,
    showtabs=false,
    tabsize=2,
    frame=none                              %set Frame
}

\begin{document}
\maketitle
\tableofcontents{}
\pagebreak

\section{Page Setup}

\section{Comments}
Comments are lines parts of code that are not compiled.\\
This can be done as a line or as an environment.

\lstset{style=basicstyle}
\begin{lstlisting}[title=Comment by Line]
  % This line would not show up on the compiled PDF.
\end{lstlisting}

\begin{lstlisting}[title=Comment by environment]
  \begin{comment}
    This line would not show up on the compiled PDF.
    This would not be in the final document either.
    Nor would this one.
  \end{comment}
\end{lstlisting}

\section{Preamble}

\section{Text Formatting}
\subsection{Font Styles}
\begin{tabular}{c c c}
  Type & Format & Result\\
  \hline
  \textbf{Bold} & \verb|\textbf{Text}| & \textbf{Text}\\
  \textbf{Italics} & \verb|\textit{Text}| & \textit{Text}\\
  \textbf{Underline} & \verb|\underline{Text}| & \underline{Text}
\end{tabular}\\
\\
\textbf{Source}
\begin{itemize}
  \item \url{https://latex-tutorial.com/symbols/text-formatting/}
\end{itemize}
\textbf{Further Study}
\begin{itemize}
  \item \url{https://www.overleaf.com/learn/latex/Bold%2C_italics_and_underlining}
  \item \url{https://latex-tutorial.com/changing-font-style/}
\end{itemize}

\subsection{Font Color}
\url{https://www.overleaf.com/learn/latex/Using_colours_in_LaTeX}
%KIR: Requires Development

\subsection{Font Sizes}
\begin{tabular}{c c c}
    Type & Format & Result\\
    \hline
    tiny & \verb|{\tiny Text}| & {\tiny Text}\\
    scriptsize & \verb|{\scriptsize Text}| & {\scriptsize Text}\\
    footnotesize & \verb|{\footnotesize Text}| & {\footnotesize Text}\\
    small & \verb|{\small Text}| & {\small Text}\\
    normalsize (default) & \verb|{\normalsize Text}| & {\normalsize Text}\\
    large & \verb|{\large Text}| & {\large Text}\\
    Large & \verb|{\Large Text}| & {\Large Text}\\
    LARGE & \verb|{\LARGE Text}| & {\LARGE Text}\\
    huge & \verb|{\huge Text}| & {\huge Text}\\
    Huge & \verb|{\Huge Text}| & {\Huge Text}\\
\end{tabular}\\
\\
\textbf{Source}
\begin{itemize}
  \item \url{https://texblog.org/2012/08/29/changing-the-font-size-in-latex/}
  \item \url{https://latex-tutorial.com/symbols/text-formatting/}
\end{itemize}

\subsection{Verbatim}
\begin{itemize}
  \item \url{https://www.overleaf.com/learn/latex/Code_listing}
\end{itemize}


\section{Symbols}
\subsection{Degrees Symbol}
The \verb|\degree| command is provided by the \verb|gensymb| package, so if you add:\\
\\
\verb|\usepackage{gensymb}|\\
\\
to your preamble, that should enable the command.\\
\\
Another alternative is the \verb|\textdegree| command, which is provided by the \verb|textcomp| package. And finally, \verb|$^{\circ}$| is another way of obtaining roughly the right symbol.


\section{Tables}
Tables are Included in \LaTeX ~there are no packages are required to be used.
\subsection{Table with no Borders}
\begin{multicols}{2}
\lstset{style=basicstyle}
\begin{lstlisting}[]
\begin{center}
  \begin{tabular}{ c c c }
    cell1 & cell2 & cell3\\
    cell4 & cell5 & cell6\\
    cell7 & cell8 & cell9
  \end{tabular}
\end{center}
\end{lstlisting}
\columnbreak
\begin{center}
  \begin{tabular}{ c c c }
    cell1 & cell2 & cell3\\
    cell4 & cell5 & cell6\\
    cell7 & cell8 & cell9
  \end{tabular}
\end{center}
\end{multicols}

\subsection{Table with Borders}
\begin{multicols}{2}
\lstset{style=basicstyle}
\begin{lstlisting}[]
\begin{center}
  \begin{tabular}{|c|c|c|}
    \hline
    cell1 & cell2 & cell3\\
    cell4 & cell5 & cell6\\
    cell7 & cell8 & cell9\\
    \hline
  \end{tabular}
\end{center}
\end{lstlisting}
\columnbreak
\begin{center}
  \begin{tabular}{|c|c|c|}
    \hline
    cell1 & cell2 & cell3\\
    cell4 & cell5 & cell6\\
    cell7 & cell8 & cell9\\
    \hline
  \end{tabular}
\end{center}
\end{multicols}

\subsection{Table with Double Borders}
\begin{multicols}{2}
\lstset{style=basicstyle}
\begin{lstlisting}[]
\begin{center}
  \begin{tabular}{||c c c||}
    \hline
    cell1 & cell2 & cell3\\ [0.5ex]
    \hline \hline
    cell4 & cell5 & cell6\\
    \hline
    cell7 & cell8 & cell9\\
    \hline
    cell10 & cell11 & cell12\\ [1ex]
    \hline
  \end{tabular}
\end{center}
\end{lstlisting}
%\columnbreak
\begin{center}
  \begin{tabular}{||c c c||}
    \hline
    cell1 & cell2 & cell3\\ [0.5ex]
    \hline \hline
    cell4 & cell5 & cell6\\
    \hline
    cell7 & cell8 & cell9\\
    \hline
    cell10 & cell11 & cell12\\ [1ex]
    \hline
  \end{tabular}
\end{center}
\end{multicols}

\noindent\textbf{Source}
\begin{itemize}
  \item \url{https://www.overleaf.com/learn/latex/Tables}
\end{itemize}


\section{Columns}
Include the following in the preamble:
\lstset{style=basicstyle}
\begin{lstlisting}[title=Preamble]
\usepackage{multicol}         %Package for columns
\setlength{\columnsep}{1cm}   %Initialize spacing between columns
\end{lstlisting}

\subsection{Column Separation}
\begin{lstlisting}[]
\begin{multicols}{2}
  [
  \noindent \textbf{First Section}\\
  All things are subject to decay. And when fate summons, Monarchs must obey.
  ]
  \blindtext
\end{multicols}
Thanks to the genius of the authors of life itself this occurs and will continue until the cycle ends.
\end{lstlisting}

\begin{multicols}{2}
  [
  \noindent \textbf{First Section}\\
  All things are subject to decay. And when fate summons, Monarchs must obey.
  ]
  \blindtext
\end{multicols}
Thanks to the genius of the authors of life itself this occurs and will continue until the cycle ends.

\noindent\textbf{Source}
\begin{itemize}
  \item \url{https://www.overleaf.com/learn/latex/Multiple_columns}
\end{itemize}

\section{Tikz}
\subsection{circuitikz}
\textbf{Source}
\begin{itemize}
  \item \url{https://www.overleaf.com/learn/latex/LaTeX_Graphics_using_TikZ%3A_A_Tutorial_for_Beginners_(Part_4)%E2%80%94Circuit_Diagrams_Using_Circuitikz}
\end{itemize}

\section{Including Files}
\subsection{Image Files}
\subsection{Coding Files}
\lstset{style=basicstyle}
\begin{lstlisting}[title=Function: Graph::isCycle()]
bool Graph::isCycle() {   //similar to DFS
for(int i = 0; i < size; i++) {
  parents[i] = i;
  colors[i] = i;
  colors[i] = 'W';
}
int t = 0;

for(int i =0; i < size; i++) {
  //nodes are either White or Black in here
  if(colors[i] == 'W'){
    //color[i] = 'G';
    bool res = isCycleVisit(i, t);
    if(res)
      return res;
  }//if
}//for
return false;
}
\end{lstlisting}
\textbf{Source}
\begin{itemize}
  \item \url{https://www.youtube.com/watch?v=ByduYnAu2jM}
  \item \url{https://www.youtube.com/watch?v=-trhlQToLiU}
  \item \url{https://www.overleaf.com/learn/latex/Code_listing}
\end{itemize}

\section{Matricies}
\lstset{style=blankstyle}
\begin{center}
	\begin{tabular}{c c c}
	Type & \LaTeX Markup & Renders As\\
	\hline
	Plain & \begin{lstlisting}
		\begin{matrix}
			1 & 2 & 3\\
			a & b & c
		\end{matrix}
		\end{lstlisting} & $\begin{matrix}
			1 & 2 & 3\\
			a & b & c
		\end{matrix}$\\
	\hline
	Parentheses; round brackets & \begin{lstlisting}
		\begin{pmatrix}
			1 & 2 & 3\\
			a & b & c
		\end{pmatrix}
		\end{lstlisting} & $\begin{pmatrix}
			1 & 2 & 3\\
			a & b & c
		\end{pmatrix}$\\
	\hline
	Brackets; square brackets & \begin{lstlisting}
		\begin{bmatrix}
			1 & 2 & 3\\
			a & b & c
		\end{bmatrix}
		\end{lstlisting} & $\begin{bmatrix}
			1 & 2 & 3\\
			a & b & c
		\end{bmatrix}$\\
	\hline
	Braces; curly brackets & \begin{lstlisting}
		\begin{Bmatrix}
			1 & 2 & 3\\
			a & b & c
		\end{Bmatrix}
		\end{lstlisting} & $\begin{Bmatrix}
			1 & 2 & 3\\
			a & b & c
		\end{Bmatrix}$\\
	\hline
	Pipes & \begin{lstlisting}
		\begin{vmatrix}
			1 & 2 & 3\\
			a & b & c
		\end{vmatrix}
		\end{lstlisting} & $\begin{vmatrix}
			1 & 2 & 3\\
			a & b & c
		\end{vmatrix}$\\
	\hline
	Double Pipes & \begin{lstlisting}
		\begin{Vmatrix}
			1 & 2 & 3\\
			a & b & c
		\end{Vmatrix}
		\end{lstlisting} & $\begin{Vmatrix}
			1 & 2 & 3\\
			a & b & c
		\end{Vmatrix}$\\
	\hline
	\end{tabular}
\end{center}

Source: \url{https://www.overleaf.com/learn/latex/Matrices}

\section{Cases}
Source: \url{https://www.overleaf.com/learn/latex/Questions\%2FHow_to_handle_mathematical_expressions_involving_case_statements}

\end{document}
