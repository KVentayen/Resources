\documentclass[10pt, letterpaper]{article}
\usepackage[letterpaper, portrait, margin=1in]{geometry}   %For page Setup
\usepackage[utf8]{inputenc}
\usepackage{amssymb, amsmath}               %For Equations and Formulas
\usepackage{comment}                        %For Commenting
\usepackage{hyperref}                       %For Hyperlinks
\usepackage{listings}                       %For Coding Examples
\usepackage[table]{xcolor}                  %For Coloring Tables
\usepackage{xcolor}                         %For Color Associated with Coding Examples
\usepackage{multicol}                       %For Making Multiple Columns
\usepackage{multirow}                       %Allows for multiple cells in one row in a table
\usepackage{cancel}						%Allow for diagonal strikthroughs
\usepackage{graphicx, epstopdf}                       %Converts eps files to pdf
\epstopdfsetup{update}

\title{Math Notes}
\author{K}
\date{March 6, 2020}

\usepackage{natbib}
\usepackage{graphicx}

\hypersetup{                                %Setup for Hyperlink Colors
    colorlinks=true,
    linkcolor=blue,                         %For Hyperlinked Text
    filecolor=magenta,                      %For Text that Hyperlinks to other Files
    urlcolor=cyan,                          %For Hyperlinked Printed URLs
}



\begin{document}

\begin{comment}
\begin{titlepage}
    %\titlepage
    \maketitle
\end{titlepage}
\end{comment}

\maketitle

\tableofcontents{}
\section{First Order Differential Equations}
\subsection{Separable Differential Equations}

\begin{align*}
\frac{dy}{dx} &= G(x) \cdot H(y) \Bigm\lvert \cdot dx\\
dy &= G(x) \cdot H(y) dx \Bigm\lvert \cdot \frac{1}{H(y)}\\
\frac{dy}{H(y)} &= G(x) dx\\
\int \limits \frac{dy}{H(y)} &= \int \limits G(x) dx\\
h(y) + C_1 &= g(x) + C_2\\
h(y) &= g(x) + C
\end{align*}

\textbf{Example}
\begin{align*}
\frac{dy}{dx} &= y \sin x\\
dy &= y \sin x (dx)\\
\frac{dy}{y} &= \sin x dx\\
\int \limits y^{-1} dy &= \int \limits \sin x dx\\
\ln |y| + C &= -\cos x + C\\
\ln |y| &= -\cos x + C\\
y &= e^{-\cos x + C}\\
&= e^{-\cos x}e^C\\
&= \frac{1}{e^{\cos x}} \cdot e^C\\
&= \frac{e^C}{e^{\cos x}}\\
& ~~~~~ D = e^C\\
&= \frac{D}{e^{\cos x}}
\end{align*}

\subsection{Linear First Order Differential Equations Homogeneous Differential Equation}
% KIR: Needs Example

$y'(x) + p(x)y(x) = q(x)$\\
$~~~~~$ $p(x)$, $q(x)$ given $q(x)=0$, then $y'+p(x)y = 0$

\begin{itemize}
	\item Linear because all terms are to the power of 1
\end{itemize}

\begin{align*}
y'+y^2 = 0 &\rightarrow \text{non-linear}\\
y'+y = 0 &\rightarrow \text{linear}
\end{align*}

\subsection{Method of Integrating Factor}
$\rho = e^{\int \limits p(x) dx}$

\begin{align*}
y'e^{\int \limits p(x) dx} + P(x)ye^{\int \limits p(x) dx} &= q(x)e^{\int \limits p(x) dx}\\
\frac{d}{dx}(y \cdot e^{\int \limits p(x) dx}) &= y'e^{\int \limits p(x) dx} \cdot \frac{d}{dx} (\int \limits p(x) dx)\\
&= y'e^{\int \limits p(x) dx} + ye^{\int \limits p(x) dx} p(x)\\
\frac{d}{dx}(y \cdot e^{\int \limits p(x) dx}) &= q(x) \cdot e^{\int \limits p(x) dx} \Bigm\lvert \cdot dx \int \limits\\
\int \limits \frac{d}{dx} (ye^{\int \limits p(x) dx})dx &= \int \limits q(x) \cdot e^{\int \limits p(x) dx}\\
y \cdot e^{\int \limits p(x) dx} &= \int \limits (q(x) e^{\int \limits p(x) dx})dy + e  \cdot \frac{1}{e^{\int \limits p(x) dx}}\\
y &= (\int \limits (q(x) \cdot e^{\int \limits p(x) dx})dx + C)e^{-\int \limits p(x) dx}
\end{align*}

\subsubsection{Method of Substitution}
\begin{enumerate}
	\item
\begin{align*}
y' &= f(ax + by + c)\\
&a, b, c \text{given constants}\\
&f \text{given functions}\\
u &= ax + by + c\\
\frac{du}{dy} &= \frac{d}{dx}(ax+by+c)\\
\frac{du}{dy} &= a +b\frac{dy}{dx} + 0\\
&\rightarrow \frac{dy}{dx} = \frac{\frac{du}{dx}-a}{b}\\
\frac{\frac{du}{dx}-a}{b} &= f(u)\\
\frac{du}{dx} &= bf(u)+a \Bigm \lvert \cdot dx\\
du &= (bf(u) + a)dx\\
\frac{du}{bf(u) + a} &= dx\\
\int \limits \frac{du}{bf(u) + a} &= \int \limits dx\\
F(u) &= x+C\\
F(ax+by+C) &= x + C
\end{align*}

\textbf{Example}
\begin{align*}
\frac{dy}{dx} &= (x+y+3)^2\\
u &= x+y+3\\
\frac{du}{dx} &= \frac{d}{dx}(x+y+3)\\
\frac{du}{dx} &= 1 + \frac{dy}{dx}\\
\frac{dy}{dx} &= \frac{du}{dx} - 1\\
\frac{du}{dx} - 1 &= (u)^2\\
\frac{du}{dx} &= 1 + (u)^2 \Bigm \lvert \cdot dx\\
du &= (1+u^2)dx \Bigm \lvert \frac{1}{1+u^2}\\
\frac{du}{1+u^2} &= dx\\
\int \limits \frac{du}{1+u^2} &= \int \limits dx\\
\tan^{-1} u &= x+C\\
\tan(\tan^{-1} u) &= \tan(x+C)\\
u &= \tan(x+C),& x+y+3 &= \tan(x+C)\\
y &= \tan(x+C)-x-3
\end{align*}

\item
\begin{align*}
y' &= f(\frac{y}{x})\\
u &= \frac{y}{x}, y = u \cdot x, \frac{dy}{dx} = \frac{du}{dx} x + 1 \cdot u\\
\frac{du}{dx} \cdot x + u &= f(u) \rightarrow \frac{du}{dx} x = f(u)-u \Bigm \lvert \cdot dx\\
du \cdot x &= (f(u)-u)dx \Bigm \lvert \frac{1}{x(f(u)-u)}\\
\frac{du}{f(u)-u} &= \frac{dy}{x}\\
F(u) &= \ln|x|+C\\
F(\frac{y}{x}) &= \ln|x|+C
\end{align*}

\begin{align*}
2xy \frac{dy}{dx} &= 4x^2 + 3y^2\\
2xy \frac{dy}{dx} &= 4x^2 + 3y^2 \Bigm \lvert \frac{1}{x^2}\\
2 \frac{y}{x} \cdot \frac{dy}{dx} &= 4 + 3(\frac{y}{x})^2\\
\frac{dy}{dx} &= \frac{4+3 (\frac{y}{x})}{2(\frac{y}{x})}\\
u &= \frac{y}{x}, y = u \cdot x, \frac{dy}{dx}=\frac{du}{dx}x+u\\
\frac{du}{dx} &= \frac{4+3u^2}{2u} -u\\
&= \frac{4+3u^2-2u^2}{2u}\\
&= \frac{4+u^2}{2u}\\
x \cdot \frac{du}{dx} &= \frac{4+u^2}{2u} \Bigm \lvert \cdot dx\\
x \cdot du &= \frac{4+u^2}{2u} \cdot dx \Bigm \lvert \cdot \frac{1}{x\cdot \frac{4\cdot u^2}{2u}}\\
\int \limits \frac{2u}{4+u^2} du &= \int \limits \frac{dx}{x}\\
&~~~~~ z=4+u^2\\
&~~~~~ dz=2udu\\
\int \limits \frac{dz}{z} &= \int \limits \frac{dx}{x}\\
\ln|z| &= \ln|x| + C\\
e^{\ln|z|} &= e^{\ln|x|+C} = e^{\ln|x|}e^{C}\\
|z| &= |x|e^C\\
z &= e^C \cdot x=\pm e^C \cdot x\\
&~~~~~ A=\pm e^C\\
z &= Ax\\
4+u^2 &= Ax\\
4+(\frac{y}{x})^2 &= Ax ~~~~~ (\text{general solution})\\
(\frac{y}{x})^2 &= Ax-4\\
\frac{y}{x} &= \pm \sqrt{Ax-4}\\
y &= \pm x \sqrt{Ax-4} ~~~~~ (\text{explicit form})
\end{align*}
\end{enumerate}

Two Types of U-Substitution
\begin{enumerate}
	\item $y' = f(ax+by+c)$
	\item $y' = f(\frac{x}{y}$
\end{enumerate}

\subsection{Exact Equations}
\begin{align*}
dF &= M(x,y)dx +N(x,y)dy =0\\
&~~~~~ dF(x,y) = 0\\
dF(x,y) &= 0\\
dF &= \frac{\partial F}{\partial x} dx + \frac{\partial F}{\partial y} dy = 0\\
[\frac{\partial F}{\partial x} = N(x,y)] & [\frac{\partial F}{\partial y} = M(x,y)]\\
\frac{\partial}{\partial y} [\frac{\partial F}{\partial x} = N(x,y)], & \frac{\partial}{\partial x} [\frac{\partial F}{\partial y} = M(x,y)]\\
\frac{\partial^2 F}{\partial y dx} = \frac{\partial N}{\partial y}, & \frac{\partial^2 F}{dx dy} = \frac{\partial M}{\partial x}\\
\frac{\partial M}{\partial x} &= \frac{\partial N}{\partial y}\\
\\
\frac{\partial F}{\partial x} &= N \Bigm \lvert \cdot dx \int \limits\\
\int \limits \frac{\partial F}{\partial x} &= \int \limits N(x,y) dx\\
F(x,y) &= \int \limits N(x,y)dx + g(y)\\
\frac{\partial F}{\partial y} &= \frac{\partial}{\partial y} \int \limits N(x,y) dx + g'(y) = M(x,y)\\
\Rightarrow g(y)\\
F(x,y) &= \int \limits N(x,y)dx + g(y)\\
&(\text{equation from earlier step})
\end{align*}

\textbf{Example}
\begin{align*}
(6xy-y^3)dx + (4y+3x^2-3xy^2)dx &= 0\\
M(x,y) + N(x,y) &= 0\\
\frac{\partial M}{\partial y}(6xy - y^3) &= 6x-3y^2\\
\frac{\partial N}{\partial x} &= \frac{\partial}{\partial x} (4y+3x^2-3xy^2) = 0+6x-3y^2\\
\\
dF &= \frac{\partial F}{\partial x} dx + \frac{\partial F}{\partial y} dy\\
\\
\frac{\partial F}{\partial x} &= 6xy-y^3, \frac{\partial F}{\partial y} = 4y+3x^2-3xy^2\\
\int \limits \frac{\partial F}{\partial x} dx &= \int \limits (6xy-y^3)dx\\
F(x,y) &= 6\frac{x^2}{2} \cdot y-y^3x+g(y)\\
\frac{\partial F}{\partial y} &= 3x^2 \cdot 1 - x \cdot 3y^2 + g'(y)\\
&\text{Remember: } \frac{\partial F}{\partial y} = 4y+3x^2-3xy^2\\
3x^2-3xy^2+g'(y) &= 4y+3x^2-3xy^2\\
g'(y) &= 4y\\
\int \limits g'(y) dy &= \int \limits 4y dy\\
g(y) &= \frac{4y^2}{2} + C\\
&\text{Remember: } F(x,y) = 6\frac{x^2}{2} \cdot y-y^3x+g(y)\\
F(x,y) &= 3x^2 \cdot y -y^3x+2y^2+C\\
\\
\int \limits dF &= 0, F(x,y)=C_1\\
\\
3x^3y - y^3x + 2y^2 &= C_1\\
y(x)
\end{align*}

\section{Bernoulli Equation}
$y'(x)+P(x)y(x) = Q(x)y^n(x)$\\
$~~~~~$ $P, Q$-given, $n \neq 1$\\
\\
linear $y' + p(x)y = q(x)$ by integrating factor $\rho = e^{\int \limits p(x) dx}$\\
\\
dividing by $y^n$ to obtain\\
\\
\begin{align*}
y^{-n}y' + P(x)y^{1-n} &= Q(x)\\
z &= y^{1-n}, (z(x))' = (y(x)^{1-n})'\\
\frac{dz}{dx} &= (1-n)y^{1-n-1} \frac{dy}{dx}\\
&= (1-n) \cdot y^{-n} \frac{dy}{dx}\\
y^{-n}y'(x) &= \frac{dz}{dx} \cdot \frac{1}{1-n}
\end{align*}

\textbf{Example}\\
\begin{align*}
y'-\frac{3}{2x}y &= \frac{2x}{y}\\
P(x) &= -\frac{3}{2x}, Q=2x, n=-1\\
\\
y'-\frac{3}{2x}y &= \frac{2x}{y} \Bigm \lvert \cdot y\\
&~~~~~ z=y^{1-n}\\
yy'-\frac{3}{2x}y^2 &= 2x\\
z &= y^2, z'(x)=2y \cdot y'(x)\\
&~~~~~ \rightarrow y \cdot y'(x) = \frac{z(x)}{z}\\
\\
\frac{z'(x)}{z} - \frac{3}{2x} z &= 2x\\
&~~~~~ \rightarrow z'(x) - \frac{3}{2} z = 4x\\
\\
\rho &= e^{\frac{-3}{2}dx}\\
&= e^{-e \ln x}\\
&= e^{\ln x^{-3}}\\
&= x^{-3}\\
\\
z'(x)-\frac{3}{x}z &= 4x \Bigm \lvert \cdot \rho = x^{-3}\\
x^{-3} \cdot z'(x) - 3x^{-4} z(x) &= 4x^{-2}\\
check &\rightarrow \frac{d}{dx} (x^{-3} \cdot z) =4x^{-2} & \frac{d}{dx}(x^{-3}\cdot z) &= z'x^{-3}-3x^{-4}z\\
& & &= x^{-3}z' + z(-3)\cdot x^{-4}\\
\int \limits \frac{d}{dx} (x^{-3}z) dx &= \int \limits 4x^{-2} dx\\
x^{-3}z &= -\frac{4}{x} + C\\
z &= -4x^2 + Cx^3\\
\\
y^2 = 4x^2+Cx^3\\
y=\pm\sqrt{-4x^2+Cx^3}
\end{align*}

\section{Second Order Ordinary Differential Equations}
$F(y'', y', y, x)=0, y=y(x)$\\
\subsection{Type I} $F(y''(x), y'(x), x)=0$\\
Function of $x$
\begin{align*}
y'(x)=P(x), y''(x) &= (y'(x))'=P'(x)\\
&\rightarrow F(p'(x), p(x), x) = 0\\
&\rightarrow p(x) = f(x, C_1)\\
&~~~~~ y'=f(x, C_1) \Bigm \lvert \cdot dx \int \limits\\
&\rightarrow \int \limits y'(x) dx = \int \limits f(x, C_1) dx\\
&~~~~~ y(x) = \int \limits f(x, C_1) dx +C_2
\end{align*}

\textbf{Example}\\
\begin{align*}
xy''+2y' &= 6x, y(x)=?\\
xp'(x)+2p &= 6x\\
\\
y'+p(x)y &= q(x) \Bigm \lvert &\cdot \rho &= e^{\int \limits \frac{2}{x} dx}\\
& & &= e^{2 \ln x}\\
& & &= x^2\\
p'(x)x^2 + \frac{2x^2}{x}p(x) &= 6x^2\\
\frac{d}{dx} (x^2 \cdot p(x)) &= p'(x)x^2 +2xp\\
& \rho \cdot p(x)\\
\frac{d}{dx} (p(x), y(x)) &= (p(x), y(x))'\\
\frac{d}{dx} (x^2 \cdot p(x)) &= 6x^2 \Bigm \lvert \cdot dx \int \limits\\
\int \limits \frac{d}{dx} (x^2 \cdot p(x)) dx &= \int \limits 6x^2 dx\\
x^2p &= \frac{6x^2}{3} +C,& p(x) &= 2x + \frac{C_1}{x^2}\\
& & &= y'(x)\\
&\rightarrow \int \limits y'(x) dx = \int \limits (2x+C_1x^{-2})dx\\
&~~~~~ y(x) = \frac{2x^2}{2} - \frac{C_1}{y} + C_2
\end{align*}

\subsection{Type II} $F(y''(x), y'(x), y(x)) = 0$\\
Function of $y$
\begin{align*}
y'(x) = p(y), y'' &= (y'(x))'\\
&= (P(y))''\\
&= \frac{dP}{dy} \cdot \frac{dy}{dx}\\
&= \frac{dp}{dy} p
\end{align*}

\textbf{Example}
\begin{align*}
yy''(x) &= (y'(x))^2\\
y\frac{dp}{dy} \cdot p &= p^2, p \neq 0\\
y\frac{dp}{dy},& \frac{dp}{p} = \frac{dy}{y}, \int \limits \frac{dp}{p} = \int \limits \frac{dy}{y}\\
&\rightarrow \ln|p| = \ln|y| + C_1\\
&~~~~~ e^{\ln|p|} = e^{\ln|y| + C_1} = e^{\ln|x|} \cdot e^{C_1}\\
&~~~~~ |p| = |y| \cdot e^{C_1}, p = \pm e^{C_1} \cdot y\\
&~~~~~~~~~~ A_1 = \pm e^{C_1}\\
&~~~~~ p=A_1 y\\
&\rightarrow y'(x) = A_1 y\\
&~~~~~ \frac{dy}{dx} = A_1 y\\
&~~~~~ \frac{dy}{y} = A_1 dx\\
&~~~~~ \int \limits \frac{dy}{y} = \int \limits A_1 dx\\
&\rightarrow \ln|y| = A_1 x + C_2\\
&~~~~~ e^{\ln|y|} = e^{A_1 x + C_2}\\
&~~~~~ |y| = e^{A_1 x} e^{C_2}\\
&\rightarrow y = \pm e^{C_2}e^{A_1 x}\\
&~~~~~~~~~~ A_2 = \pm e^{C_2}\\
&~~~~~ y=A_2 e^{A_1 x}
\end{align*}

\subsection{Inital Value Problem}
(1) $\frac{dy}{dx} = f(x,y), y(x_0)=y_0$\\
\underline{Then} If $f(x,y), \frac{\partial f}{\partial y}$ (continnuous function) on $R[a \leq x \leq b, c \leq y \leq d]$ there exists wuch interval $I \in [a,b]$ also $x_0 \in I$, where the inital value problem (1) has unique solution\\
\\
% KIR: Insert Image of Graph Here
Rectangle (R) is designated by the chosen value (can be very large or small)

\begin{align*}
y''+p(x)y'+g(x)y &= f(x), x \in (a,b)\\
p(x), q(x), f(x)& - given y(x)?\\
y(x_0) = y_0,& y'(x_0)=y_1
\end{align*}
\textbf{Theorem 1} if $p(x), q(x), f(x)$\\
- continuous on $(a, b), x_0 \in (a,b)$ then (2) has unique solution\\

\section{Linear Second Order Differential Equations}
\subsection{Initial Value Problem}

\begin{align*}
y'' + p(x)y' + q(x)y &= f(x), x \in (a,b)\\
p(x), q(x), f(x) &- \text{given} y(x)?\\
y(x_0) = y_0&, y'(x_0) = y_1
\end{align*}

\subsubsection{Theorem 1}
if $p(x), q(x), f(x)$\\
- continuous on $(a,b), x_0 \in (a,b)$ then (2) has unique solution

\begin{align*}
f(x) \equiv 0, y'' + p(x)y' + g(x)y = 0\\
\text{homogeneous equation}
\end{align*}

\begin{align*}
0= \text{homogeneous equation}\\
0\neq \text{non homogeneous equation}
\end{align*}

\begin{align*}
y_1(x), y_2(x) &- \text{solutions of (3)}\\
y(x) &= C_1y_1(x) + C_2y_2(x)\\
&- \text{is also solution of (3) linear combination, where }C_1, C_2 - \text{some constant}
\end{align*}

\begin{align*}
(C_1y_1 + C_2y_2)'' + p(x)(C_1y_1 + C_2y_2) &= 0\\
C_1y_1'' + C_2y_2 + p(x)(C_1y_1' + C_2y_2') + q(x)(C_1y_1 + C_2y_2) &= 0\\
C_1(y_1'' + p(x)y_1' + q(x)y_1) + C_2(y_2'' + p(x)y_2' + q(x)y_2) &= 0\\
\\
y_1'' + p(x)y_1' + q(x)y_1 &= 0\\
y_2'' + p(x)y_2' + q(x)y_2 &= 0
\end{align*}

Definition $y_1(x), y_2(x)$ - linear independent on $(a, b)$ if $C_1y_1(x) + C_2y_2(x) = 0$ if $C_1 = C_2 = 0$\\
\\
\begin{align*}
y_1 &= \Bigl ( -\frac{C_2}{C_1} \Bigl) y_2(x) & y_1 &= K \cdot y_2\\
& & \frac{y_1}{y_2} & K
\end{align*}

\textbf{Example 1}\\
$y_1 = \sin x, y_1 = \cos x$\\
$\frac{\sin x}{\cos x} \neq K$\\
linear independent $\rightarrow$ ratio is not equal to constant

\textbf{Example 2}\\
$y_1 = \sin 2x, y_2 = \sin x \cos y$\\
$\frac{y_1}{y_2} = \frac{\sin 2x}{\sin x \cos x} = \frac{2 \sin x \cos x}{\sin x \cos x} = 2$

\section{Second Order Linear Differential Equations}
\begin{enumerate}
	\item $y'' + p(x)y' + q(x)y = f(y)$
	\begin{itemize}
		\item non homogeneous equation
	\end{itemize}
	\item $y(x_0) = y_0, y'(x_0)=y_1$
	\begin{itemize}
		\item inital conditions
	\end{itemize}
	\item 1 + 2 $\rightarrow$ I.V.P. - has unique solution when p(x), q(x), f(x) - continuous
\end{enumerate}

\subsection{Homogeneous Equation}
\begin{equation*}
\text{(3)} f(x) = 0 \rightarrow y''+p(x)y'+q(x)y=0
\end{equation*}

\begin{align*}
y_1(x), y_2(x) &\text{- linear independent}\\
&\text{if } C_1y_1(x) + C_2y_2(x) = 0\\
&\text{if } C_1 = c_2 = 0\\
\\
&y=\frac{C_2}{C_1}y_2, k = \frac{C_2}{C_1}\\
&y_1 = ky_2
\end{align*}

\textbf{Theorem}\\
If $y_1(x), y_2(x)$ - are solutions of equation (3) then if (1) $y_1, y_2$ - linear independent on $(a, b)$ then wronskian:
\begin{equation*}
W(y_1, y_2) =
\begin{vmatrix}
y_1(x) & y_2(x)\\
y_1'(x) & y_2'(x)
\end{vmatrix}
\neq 0
\\
\text{for all } a<x<b
\end{equation*}

(2) $y_1, y_2$ - linear dependent, then
$
\begin{vmatrix}
y_1 & y_2\\
y_1' & y_2'
\end{vmatrix}
=0$
for all $x$.

\textbf{Theorem}\\
If $y_1(x), y_2(x)$ - linear independent solutions of (3), then $y=C$, $y_1+C_2y_2(x)$ -generalsolution of (3), where $C_1$, $C_2$ - arbitrary constant\\
\\
\subsection{Solving Initial Value Problem $(3) + (2)$}
\begin{enumerate}
	\item find particular $y_1(x)$, $y_2(x)$
	\begin{itemize}
		linear independence
	\end{itemize}
	\item Set up general solution of $(3)$
	\begin{itemize}
		\item $y = C_1y_1 + C_2y_2$
	\end{itemize}
	\item Satisfy Initial Condition $(2)$
	\begin{itemize}
		\item 1.e subset solution $y=C_1y_1 + C_2y_2$
	\end{itemize}
\end{enumerate}

\begin{equation*}
\begin{matrix}
y(x_0) = C_1y_1(x_0) + C_2y_2(x_0) = y_0\\
y'(x_0) = C_1y_1'(x_0) + C_2y_2'(x_0) = y_1
\end{matrix}
\Bigm \} C_1, C_2
\end{equation*}

\begin{equation*}
\Bigm \{
\begin{matrix}
a_{11}C_1 + a_{12}C_2 =d_1\\
a_{21}C_1 + a_{22}C_2 =d_2
\end{matrix}
\Bigm | C_1, C_2
\end{equation*}

\begin{equation*}
\begin{vmatrix}
a_{11} & a_{12}\\
a_{21} & a_{22}
\end{vmatrix}
\neq 0
\end{equation*}

$a_{11}, a{12}, a{21}, a{22}, d_1, d_2$ are given constants

\section{Method of Elimination}
$C_1, C_2$ -?\\

\begin{align*}
&\Bigm \{
\begin{matrix}
5C_1 + 3C_2 = 1\\
C_1 - 2C_2 = 8
\end{matrix}
\\
&\Bigm \{
\begin{matrix}
	5C_1 + 3C_2 = 1\\
	C_1 - 2C_2 = 8
\end{matrix}
\Bigm |
\begin{matrix}
	\\
	\cdot (-5)
\end{matrix}
\\
(+) &\Bigm \{
\begin{matrix}
	5C_1 + 3C_2 = 1\\
	-5C_1 + 10C_2 = 8
\end{matrix}
\\
&0 + 13C_2 = -39\\
&\rightarrow 13C_2 = -39\\
&C_2 = -3\\
\\
&C_1 -2(-3) = 8\\
&C_1 = 2
\end{align*}

\section{Method of Substitution}
\begin{align*}
5(2C_2+8)+3C_2 &= 1\\
13C_2 + 40 &=1\\
13C_2 &= -39\\
C_2 &= -3\\
\\
C_1 &= 2C_2 + 8\\
&= 2(-3) + 8 = 2\\
C_1 - 2 \cdot (-3) &= 8\\
C_1 = 2
\end{align*}

\section{Homogeneous Equation with Constant Coefficients}
$ay'' + by' + cy = 0$; $a, b, c$ - given constant\\
\begin{align*}
y &= e^{rx}, r-\text{constant}\\
y' &= e^{rx} \cdot r\\
y'' &= e^{rx} \cdot r \cdot r = e^{rx} \cdot r^2\\
\\
a(e^{rx}r^2) + b(e^{rx}r) + c(e^{rx}) &= 0 \Bigm | \frac{1}{e^{rx}}\\
ar^2 + br + c &= 0\\
r_{1, 2} = \frac{-b \pm \sqrt{b^2 - 4ac}}{2a}
\end{align*}

\begin{align*}
(1)~~~r&=r_1, r=r_2 \text{- distinct } r_1 \neq r_2\\
y_1 &= e^{r_1x}, y_2 = e^{r_2x}\\
&\text{-linear independent}
\end{align*}

\begin{align*}
W(y_1, y_2) &=
\begin{vmatrix}
e^{r_1x} & e^{r_2x}\\
r_1e^{r_1x} & r_2e^{r_2x}
\end{vmatrix}\\
&= r_2e^{r_1x}e^{r_2x} - re^{r_1x}e^{r_2x}\\
&= C_1e^{r_1x} + C_2e^{r_2x}\\
&~~~~~\text{- general solution}
\end{align*}

\textbf{Example 1}\\
\begin{align*}
y''-5y'+by = 0 ~~~~~ y=e^{rx}\\
r^2 -5r +6 =0\\
r_1=3, r_2=2\\
y_1=e^{3x}, y_2=e^{2x}
\end{align*}

\textbf{Example 2}
\begin{align*}
y'' +2y' =0 \leftarrow y=e^{rx}\\
r^2 +2r =0, r(r+2)=0\\
r=0, r=-2\\
y_1 = e^{0 \cdot x} = 1, y_2=e^{-2x}\\
\rightarrow y =C_1 +C_2 \cdot e^{-2x}\\
\end{align*}

\begin{align*}
ay`` +by` +cy =0 \leftarrow e^{rx} =y\\
ar^2 +br +c=0\\
r=r_1, r=r_2, r_1=r_2\\
a(r-r_1)^2 =0\\
a(r^2 -2r_1r +ar^2) =0\\
b=(-2ar), c=(ar^2)\\
y_1(x) =e^{r_1x}, y_2(x) =e^{r_1x}x\\
y =C_1e^{r_1x} +C_2e^{r_2x}
\end{align*}

\section{Euler's Equation}
$ax^2y''(x) +bxy'(x) +cy(x) =0$\\
\\
\textbf{Example 2}
\begin{align*}
x^2y'' +xy' -y &=0\\
v = \ln x, &y(v)\\
y'(x) &= \frac{dy}{dv} \cdot \frac{dv}{dx} = \frac{dy}{dv} \cdot \frac{1}{x}\\
y''(x) &= \frac{x}{dx} (y'(x))\\
&= \frac{d}{dx} \Bigm( \frac{dy}{dv} \cdot \frac{1}{x} \Bigm)\\
&= \frac{d}{dx} \Bigm( \frac{dy}{dv} \Bigm) \frac{1}{x} + \frac{dy}{dv} \Bigm( -\frac{1}{x^2} \Bigm)\\
&= \frac{d}{dv} \Bigm( \frac{dy}{dv} \Bigm) \frac{dv}{dx} \cdot \frac{1}{x} - \frac{1}{x^2} \cdot \frac{dy}{dx}\\
&= \frac{d^2}{dv} \cdot \frac{1}{x} \cdot \frac{1}{x} - \frac{1}{x} \cdot \frac{dy}{dv}
\end{align*}

\begin{align*}
x^2 \Bigm( \frac{d^2y}{dy^2} \cdot \frac{1}{x} - \frac{dy}{dv} \cdot \frac{1}{x^2} \Bigm) + x \Bigm( \frac{dy}{dv} \cdot \frac{1}{x} \Bigm) -y =0\\
\frac{d^2y}{dy^2} - \frac{dy}{dv} + \frac{dy}{dv} - y\\
\frac{d^2y}{dv^2} - y = 0 \leftarrow y(v) = e^{rv}\\
r^2-1=0, r_1=1, r_2=-1
\end{align*}

\begin{align*}
y &= C_1e^v +C_2e^{-v}\\
&= C_1e^v +C_2e^{\ln x} +C_2e^{-\ln x}\\
&= C_1x + \frac{C_2}{x} = y
\end{align*}

\section{Higher Order Differential Equation}
$y^{(n)} +p_1(n)y^{(n-1)} +\cdots +p_{n-1}(x)y'(x) +p_n(x)y =f(x)$\\
$p_1, p_2, p_3, \dots, p_n$\\
\\
f-continuous function on $(a, b)$\\
\\
if $f(x) \neq 0$ then equation is non-homogeneous\\
if $f(x) = g(x)$ then equation is homogeneous\\
\\
$y_(x), y_2(x), \dots, y_n(x) -n$\\
particular linear independent solutions of homogeneous equations\\
\\
then $y(x) =C_1y_1 +C_2y_2 +\cdots +C_ny_n$\\
$C_1, C_2, \dots, C_n$ - constants\\
\\
linear independent\\
\begin{align*}
W[y_1, y_2, \dots, y_n] &\neq 0\\
W[y_1, y_2, \dots, y_n] &=
\begin{vmatrix}
y_1 & y_2 & y_3 & \cdots & y_n\\
y_1' & y_2' & y_3' & \cdots & y_n'\\
\vdots & \vdots & \vdots & \ddots & \vdots\\
y_1^{(n-1)} & y_2^{n-3} & y_3^{n-4} & \cdots & y_n^{(n-1)}
\end{vmatrix}\\
&\neq 0
\end{align*}

$S_1, S_2, S_3, S_4, \dots, S_n$\\
- linear independent\\
if $C_1f_1(x) +C_2f_2(x) +\cdots +C_nf_n(x) =0$\\
if $C_1 =C_2 =\cdots =C_n =0$\\
\\
\textbf{Problem 3}\\
\begin{align*}
f(x) =0, g(x) =\sin x, \ln(x) =e^x\\
C_1f(a) +C_2g(x) +C_3e^x =0\\
C_x \cdot 0 +C_2 \sin x + C_3 e^x =0\\
C_1=100, C_2=0, C_3=0\\
\end{align*}
\begin{align*}
\text{Determinate: }\\
W[0, \sin x, e^x] &=
\begin{vmatrix}
0 & \sin x & e^x\\
0 & \cos x & e^x\\
0 & -\sin x & e^x
\end{vmatrix}\\
&\equiv 0
\end{align*}

\textbf{Problem 17}\\
\begin{align*}
y^{(3)} -3y'' +3y' -y =0\\
y(0) =2, y'(0)=0, y''(0)=0\\
y_1 =e^x, y_2 =x\cdot e^x, y_3 =x^2e^x
\end{align*}
general solution
\begin{align*}
y &= C_1y_1 +C_2y_2 +C_3y_3\\
&= C_1e^x +C_2xe^x +C_3x^2e^x\\
y(0) &= C_1e^0 +C_20e^0 +C_30^2e^0\\
&=2\\
\\
y'(x) &= C_1e^0 +C_2(e^x +xe^x) +C_3(2xe^x +x^2e^x)\\
y'(0) &= C_1e^0 +C_2(e^0 +0) +C_3(0+0)\\
&= 0\\
\\
y''(x) &= C_1e^x +C_2(e^x +e^x +e^x) +C_3(2e^x +2xe^x +2xe^x +x^2e^x)\\
y''(0) &= C_1e^0 +C_2(e^0 +e^0 +0) +C_3(3e^0 +0 +0 +0)\\
&=0\\
\\
&\begin{cases}
C_1 =2\\
C_1 +2 =0\\
C_1 +2C_2 +2C_3 =0
\end{cases}\\
\\
&\begin{cases}
C_2 =-C_1 =-2\\
2C_3 =-C_1 -2C_2 =-2-2(-2) =2\\
C_3 =1
\end{cases}\\
\\
y &= 2e^x -2xe^x +x^2e^x
\end{align*}

(1) $y^{(n)} +p_1(x)y^{(n-1)} +\cdots +p_{n-1}\cdot y' +p_ny =0$\\
homogeneous equation\\
\\

general solution of (1):
$y =C_1y_1(x) +C_2y_2(x) +\cdots +C_ny_n$\\
$y_1, y_2, \dots, y_3$\\
-linear independent particular solutions of (1)\\
\\
(2) $y(x_0) =y_0, y'(x_0)=y_1, \cdots, y^{(n-1)} =y_{n-1}$\\
-initial conditions\\
(1) + (2) -Initial Value Problem\\
(3) $y^{(n)} +p_1(x)y^{(n+1)} +\cdots +p_{n-1}(x)y' +p_n(x) \cdot y =f(x)$\\
$f(x) \neq 0$ non-homogenous equation\\
\\
$\rightarrow$ General Solution of (3):\\
\begin{align*}
y &= C_1y_1 +C_2y_2 +\cdots +C_ny_n +y_p\\
&= (\text{General Solution of }(1)) + (\text{Particular Solution of } (3))\\
y_c &= C_1y_1 + C_2y_2 +\cdots +C_ny_n\\
&\text{Complementary Solution of (3)}
\end{align*}
\\
\begin{align*}
y &=y_c +y_p\\
&= (\text{General Solution of }(1)) + (\text{Particular Solution of }(3))
\end{align*}

$\rightarrow$Initial Value Problem for (3): (3) + (2)\\
\\
\textbf{Example 1}\\
\begin{align*}
y'' -y =12x\\
y(0) =5 ,y'(0) =7
\end{align*}

(1) homogeneous equation:
\begin{align*}
y'' -4y &= 0 \rightarrow y =e^{rx}\\
r^2 \dot e^{rx} -4 &= 0 \Bigm | \frac{1}{e^{rx}}\\
r^2 -4 &= 0 & r &= r_1 =2\\
& & r &= r_2 = -2\\
y_1 &= e^{2x}, y_2 = e^{-2x}\\
\rightarrow y &= C_1e^{2x} +C_2e^{-2x}\\
&\text{general solution}
\end{align*}

(2) non-homogeneous
\begin{align*}
y &= y_c +y_p & y_p &= 3x\\
& y_c =C_1e^{2x} +C_2e^{-2x}\\
y_p &= 3x \rightarrow 0 - 4(3x) \equiv -12x\\
y &= C_1e^{2x} +C_2e^{-2x} +3x\\
& \text{General Solution } = C_1e^{2x} +C_2e^{-2x}\\
& \text{Particular Solution } = 3x\\
\\
y(0) &= C_1e^0 +C_2e^0 +3 \cdot 0 =5\\
y(0) &= 2C_2e^0 -2C_2e^0 +3 =7\\
\\
C_1 +C_2 &= 5\\
2C_1 -2C_2 +3 &= 7\\
\\
C_1 +C_2 &= 5\\
C_1 -C_2 &= 2\\
\\
2C_1 +0 &= 7\\
\\
C_1 &= \frac{7}{2}\\
C_2 &= 5 -C_1\\
&= 5 -\frac{7}{2} = \frac{3}{2}\\
\\
y &= \frac{7}{2}e^{2x} +\frac{3}{2}e^{-2x} +3x
\end{align*}

\section{Higher Order Linear Differential Equations}
(1) $y^{(1)} +p_1y^{(n-1)} +\cdots +p_{n-1}y' +p_ny =f(x)$\\
- non homogeneous equation\\
general solution y of (1) is $y=y_c+y_p$,\\
$y_p$ - any particular solution of (1)\\
$y_c$ - complementary solution which is the general solution of homogeneous equation $[(1) \text{ if } f(x) \equiv 0]$\\
$y_c =C_1y_1 +C_2y_2 +\cdots +C_ny_n$, where $y_1, y_2, \cdots, y_n$\\
-linear independent solutions of homogeneous equations\\
\\
\textbf{Example 1}\\
\begin{align*}
y'' -4y &=-12x, y_p =3x\\
y_C &= ? \rightarrow y'' -4y =0\\
y &= C_1y_1 + C_2y_2 = C_1e^{2x} +C_2ye^{-2x}\\
y &= y_C +y_P =C_1e^{2x} +C_2e^{-2x} +3x\\
\end{align*}

(2) $y(x_0) =y_0, y'(x_0)=y_1, y''(x_0)=y_2, \dots, y^{(n-1)}(x_0) =y_{n-1}$\\
problem (1) + (2) - Initial Value Proplem (IVP) for (1)\\
$y' =f(x, y), y(x_0) =y_1$\\
$y'' +p_1y' +p_2y =f(x), y(x_0) =0, y'(x_0) =y_1$

\section{Linear Equation with Constant Coefficient}
\begin{align*}
a_0, a_1, a_2, \dots, a_n\\
y^{(n)} +a_1y^{(n-1)} +a_2y^{(n-2)} +\cdots +a_{n-1}y' +a_ny\\
= \sum_{i=1}^{n} a_{n-1} y^{(i)} &= 0\\
&y=e^{rx}\\
r^ne^{rx} +a_1r^{n-1}e^{rx} +a_2r^{n-2}e^{rx} +\cdots +a_{n-1}re^{rx} +a_ne^{rx} &= 0\\
\frac{r^ne^{rx} +a_1r^{n-1}e^{rx} +a_2r^{n-2}e^{rx} +\cdots +a_{n-1}re^{rx} +a_ne^{rx}}{e^{rx}} &= 0\\
r^n +a_1r^{n-1} +a_2r^{n-2} +\cdots +a_{n-1}r +a_n = 0\\
\text{-characteristic equation}\\
r = r_1(r-r_1)
\end{align*}

\subsection{Rules for Characteristic Equation}
(1) roots are $n$ distinct real numbers\\
\begin{align*}
r=r_1 &, r=r_2, \cdots, r=r_n\\
\rightarrow y_1 &= e^{r_1x}, y_2 = e^{r_2x}, y_3 = e^{r_3x}, \cdots, y_n = e^{r_nx}\\
&y_n = e^{r_nx}\\
y &= C_1e^{r_1x} +C_2e^{r_2x} +\cdots +C_ne^{r_nx}\\
&= \sum_{z=1}^{n} C_1e^{r_1x}
\end{align*}
(2) $r=r_s, r=r_s, \cdots, r=r_s$ for $k$ times\\
\begin{align*}
y_1 &= e^{r_2x}, y_2 = x \cdot e^{r_s \cdot x}, y_3 = x^2 \cdot e^{r_sx}, \dots, y_k=x^{k-1}e^{r_sx}\\
&\text{multiplicity of }k
\end{align*}

\textbf{Example 2}
\begin{align*}
y^{(3)} +3y'' +3y' +y &= 0 \leftarrow ye^{rx}\\
r^3e^{rx} +3r^2e^{rx} +3re^{rx} +e^{rx} &= 0\\
\frac{r^3e^{rx} +3r^2e^{rx} +3re^{rx} +e^{rx}}{e^{rx}} &= \frac{0}{e^{rx}}\\
r^3 +3r^2 +3r +1 &= 0
\end{align*}

General Rule of Algebraic Polynomials
\begin{itemize}
	\item root is such number that divides all values as an integer
\end{itemize}

Long Division\\
$(r+1)^3=0$\\
$(r+1)(r+1)(r+1)=0$\\
$r=r_1=-1, r=r_2=-1, r=r_3=-1$\\
\\
One Repeated Root of Multiplicity 3\\
$y_1=e^{-x}, y_2=xe^{-x}, y_3=x^2e^{-x}$\\
$y=C_1e^{-x} +C_2e^{-x}x +C_3e^{-x}x^2$\\
\\
\textbf{Example 3} (P. 134 \#26)\\
\begin{align*}
y^{(3)} +10y'' +25y' &=0; y(0)=3, y'(0)=4, y''(0)=5\\
&\rightarrow y=e^{rx}\\
r^3e^{ex} +10r^2e^{rx} +25re^{rx} &=0\\
\frac{r^3e^{ex} +10r^2e^{rx} +25re^{rx}}{e^{rx}} &= \frac{0}{e^{rx}}\\
\rightarrow r^3 +10r^2 +25r &= 0\\
r(r^2 +10r +25) &= 0\\
r(r+5)^2 &= 0\\
&\text{repeated root of multiplicity 2}\\
r &= r_1=0, r=r_2=-5, r=r_3=-5\\
y_1 &= e^0, y_2=e^{-5x}, y_3=e^{-5x}x\\
y &= C_1 +C_2e^5x +C_3e^{-5x}x\\
\\
y'(x) &= C_2(-5)e^{-5x} +C_3[e^{-5x}-5xe^{-5x}]\\
y''(x) &= C_2 25e^{-5x} +C_3[-5e^{-5x}-5e^{-5x} +25xe^{-5x}]\\
\\
&\text{Initial Conditions}\\
&\begin{cases}
	y(0) = C_2 +C_2e^0 +C_3 \cdot 0 =3\\
	y'(0) = -5C_2e^0 +C_3[e^0-0] =4\\
	y''(0) = 25C_2e^0 +C_3[-5e^02+0] =5
\end{cases}\\
&\begin{cases}
	C_1+C_2=3\\
	-5C_2 + C_3 = 4\\
	25C_2 -10C_3 =5
\end{cases}\\
\\
0-5C_3 &= 25\\
25C_2 &= 5+10C_3 = 5-50\\
C_2 &= \frac{1}{5}-10 = C_1\frac{4}{1}\\
C_1 &= 3-C_2
\end{align*}

\section{Complex Roots of Charatistics Equations}
Imaginary Number $\rightarrow \sqrt{-1} = i$\\
Complex Number $\rightarrow 3 \pm 4\sqrt{-1} = 3 \pm 4i$\\
Real Number $\rightarrow \text{Re}[3\pm4i] = 3$\\
Imaginary Number $\rightarrow \text{Re}[3\pm4i] = \pm4$\\
\\
Complex Number Plane\\
% KIR: Insert an image of the Complex Number Plane
\\
\begin{align*}
ay''+by'+cy &= 0, y = e^{rx}\\
ar^2+br+c &= 0\\
r_{1,2} &= \frac{b \pm \sqrt{b^2-4ac}}{2a}, b^2-4ac=0\\
r &= \alpha \pm i\beta, \sqrt{-1} = i\\
y_1=e^{(\alpha+i\beta)x} &, y_2=e^{(\alpha-i\beta)x}
\end{align*}

\subsection{Euler Formula}
\begin{align*}
e^{i\beta x} &= \cos(\beta x) + i\sin(\beta x)\\
e^{-i\beta x} &= \cos(\beta x) - i\sin(\beta x)
\end{align*}

\begin{align*}
y_1 &= e^{(\alpha+\beta i)x} &= e^{\alpha x}e^{\beta i}\\
&= e^{\alpha x} (\cos \beta x + i\sin \beta x)
\end{align*}

\begin{align*}
y &= u(x) + iw(x)\\
a(u+iw)'' + b(u+iw)' + c(u+w) &\equiv 0\\
(a'' bu + cu) + (aw'' + bw' + cw)i &\equiv 0\\
0+0 &= 0
\end{align*}

\begin{align*}
y_1 = e^{\alpha x} \cos \beta x, y_1 = e^{\alpha x} \sin \beta x\\
y = C_1 e^{\alpha x} \cos{\beta x} + C_2 e^{\alpha x} \sin \beta x\\
\end{align*}

\section{Method of Undetermined Coefficients}
\begin{align*}
(r^2 + 4)(ar^2 + br + c) &= 0\\
&\equiv 6r^4 + 5r^3 + 25r^2 + 20r + 4\\
&\equiv ar^4 + br^3 + cr^2 + 4ar^2 + 4br + 4c\\
\end{align*}

\begin{align*}
r^4:& 6 = a\\
r^3:& 5 = b\\
r^2:& 25 = c+4a\\
r:& 20 = 4b\\
r^0:& 4 = 4C \rightarrow c=1
\end{align*}

\begin{align*}
(r^2 + 4)(6r^2 + 5r + 1)=0
\end{align*}


\section{Non-Homogeneous Equation}
$y^{(n)} +p_1y^{n-1} +\cdots +p_{n-1}y' +p_ny =f(x)$\\
$f(x) \neq 0$\\
general solution $y=y_c+y_p$\\
$y_c$ - general solution of homogeneous $(f(x)\equiv0)$ equation\\
$y_p$ - any particular solution of non-homogeneous equation\\

\section{Method of Undetermined Coefficients}
\begin{align*}
(1) ~~~ f(x) &= \sum_{k-0}{n} A_kx^k\\
\rightarrow y_p &= \sum_{k=0}^{n} B_kx^k
\end{align*}

\textbf{Example 1}
\begin{align*}
y''+3y'+4y &= 3x+2\\
\rightarrow y_p &= Ax+B\\
(A+B)'' +3(Ax+B)' +4(Ax+B) &= 3x+2\\
0 +3A +4(Ax+B) &= 3x+2\\
\\
x &: 4A =3 \rightarrow A=\frac{3}{4}\\
x^0 &: 3A+4B =2\\
B &=\frac{2-3A}{4} =\frac{2-3-\frac{3}{4}}{4}\\
&= -\frac{1}{16}\\
\\
(2) ~~~ f(x) &= a \sin kx + b \cos kx\\
y_p &= A \sin kx + B \cos kx\\
&A, B undetermined
\end{align*}

\textbf{Example 2}
\begin{align*}
y'-2y &= 2 \cos x\\
\nwarrow y_p &= A\sin x +B\cos x\\
3(A\sin x+B\cos x)''+(A\sin x+B\cos x) &= 2\cos x\\
3(-A\sin x+B\cos x) &= 2\cos x\\
\\
\begin{matrix}
\cos x: -B+A-2B=2\\
\sin x: -3A-B-2A=0
\end{matrix}
\Bigm\}\\
\\
A-5B &= 2\\
-5-B &= 0\\
\\
B &= -5A\\
A+25 &= 2\\
A &= \frac{1}{13}\\
\\
B &= -\frac{5}{13}\\
y_p &= \frac{1}{13}\sin x - \frac{5}{13}\cos x\\
\\
(3) ~~~ f(x) &= e^{px}\sum_{k=0}^{n} A_kx^k, y_p=e^{px}\sum_{k=0}^{n} B_kx^k\\
&B_k\text{ - undetermined}
\end{align*}

\textbf{Example 3}
\begin{align*}
y''-4y &= 2e^{3x}, y_p=Be^{3x}\\
(Be^{3x})''-4Be^{3x} &= 2e^{3x}\\
9Be^{3x}-4Be^{3x} &= 2e^{3x}\\
\rightarrow 9B-4B &= 2, B=\frac{2}{5}
\end{align*}

\textbf{Example 4}
\begin{align*}
y''-4y &= 3x^2e^{3x}\\
y_p &= (Ax^2+B_x+C)e^{3x}
\end{align*}

\textbf{Example 5}
\begin{align*}
y''-4y &= 2\cdot e^{2x}\\
\nwarrow y_p &= A \cdot e^{2x}\\
(A \cdot e^{2x})''-4Ae^{2x} &= 2e^{2x}\\
4Ae^{2x}-4Ae^{2x} &= 2e^{2x}\\
0 &= 2\cdot e^{2x}\\
\\
y''-4y &= 0\\
r^2-4 &= 0\\
r = \pm 2\\
y_p = A_x \cdot e^{2x}
\end{align*}

\begin{align*}
y^{(n)} +\cdots +p_ny &= f(x)\\
y &= y_c +y_p
\end{align*}

\noindent$(1) ~~~ f(x) = \sum_{k=0}^{n} a_kx^n, y_p = \sum_{n=0}^n A_k+x^k$ when $r=0$ is not a root of characteristic equation\\
$~~~~~~~~~~\rightarrow$ n-degree polynomial\\
\\
$y_p = \Bigm( \sum_{k=0}^{n} a_kx^k \Bigm) x^m$\\
if $r=p$ is a root of characteristic equation of multiplicity $m$\\
\\
\\
$(2) ~~~ f(x) = e^{px} \sum_{k=0}^n a_kx^k, y_p=e^{px} \sum_{k=0}^nA_kx^k$ if $r=p$ is not a root of characteristic equation\\
\\
$y_p = \Bigm( e^{px} \sum_{k=0}^n A_kx^k \Bigm) x^m$ if $r=p$ is a root of characteristic equation of multiplicity $m$\\
\\
\\
$(3) ~~~ f(x) = e^{px} (p(x)\cos qx +Q_m(x)\sin (qx)),\\ y_p = e^px(\overline{p_k}(x)\cos(qx) +\overline{Q_k}(x)\sin(qx))$ where $k=mx(n,m)$, if $r\neq p+iq$\\
\\
$y_p = e^{px}(\overline{p_k}(x)\cos(qx) +\overline{Q_k}(x)\sin(qx))x^m$ of multiplicity m, then\\
\\
$y^{(n)} +\cdots +p_ny =f(x)$\\
$(1) ~~~ f(x) = e^{px} (p_n(x)\cos qx +Q_m(x)\sin qx)\\
y_p=e^{px} (\overline{p_k}(x)\cos qx + \overline{Q_k}(x)\sin qx)x^m\\
k=max(n,m)$\\
\\
$y_p =e^{px}(\overline{p_k}\cos qx +\overline{Q_k}(x)\sin qx)x^m\\
\rightarrow$ if $p+iq=r$ - root of characteristic equation of multiplicity ``$m$''\\
\\
$(2) ~~~ f(x)=e^{px}, P_k(x)\\
y_p=e^{px} \cdot Q_k(x), p=0$\\
\\
$y_p=e^{px}Q_n(x)x^m$\\
if $p=r$ -root of characteristic equation of multiplicity, ``$m$''\\
\\
if $f(x)$ $i$th sum of the above functions,
\begin{align*}
y^{(n)} +\cdots +p_ny = f_1(x) +f_2(x) &\rightarrow y_p =(y_1)_p +(y_2)_p\\
(1) ~~~ y_1^{(k)} +\cdots +p_ky_1 =f_1(x) &\rightarrow (y_1)_p\\
(2) ~~~ y_2^{(k)} +\cdots +p_ky_2 =f_2(x) &\rightarrow (y_2)_p
\end{align*}
\\
$(1) ~~~ f(x) =e^{px}[p_k(x)\sin(qx) +Q_m(x)\cos(qx)]\\
y_p = e^{px}[\overline{p_k}(x)\sin qx +\overline{Q_k}(x)\cos(qx)]x^m$\\
\\
$(2) ~~~ f(x) =e^{px}\cdot P_k(x), p=r, m\\
y=e^{px} \frac{1}{P_k(x)} \cdot x^m$
\\
\\
\textbf{Problem 25}
\begin{align*}
y'' +3y' +2y &= xe^{-x}-e^{2x}\\
\\
y_p &= (y_1)_p = (y_2)_p\\
\\
1. ~~~ y_1'' +3y_1' +2y_1 &= x\cdot e^{-x} =f_1(x)\\
(y_1)_p &= (ax+b)e^{-x}\\
\\
2. ~~~ y_2'' +3y_2' +2y_2 &= xe^{-2x} = f_2(x)\\
(y_2)_p &= (cx + d)e^{-2}
\end{align*}
\begin{align*}
y=e^{rx} \rightarrow y''+3y'+2y &= 0\\
r^2 +3r +2 &= 0\\
(r+1)(r+2) &= 0\\
m=1, ~~~ r &= -1\\
r &= -2
\end{align*}
\\
\textbf{Example 1}
\begin{align*}
(y_1)_p &= (ax+b)e^{-x}\\
&= (ax^2+bx)e^{-x}\\
(y_1)p' &= (2ax+b)e^{-x}-(ax^2+bx)e^{-x}\\
(y_1)p'' &= 2ae^{-x}-e^{-x}(2a+b)-(2ax+b)e^{-x}+e^{-x}(ax^2+b)\\
&= 2ae^{-x}-2e^{-x}(2ax+b)+e^{-x}(ax^2+bx)\\
&= 2ae^{-x}-2e^{-x}(ax+b)+e^{-x}(ax^2+bx)+3[(2ax+b)e^{-x}-(ax^2+bx)e^{-x}]+2(ax^2+bx)e^{-x}\\
&= x \cdot e^{-x}
\end{align*}
\\
\begin{align*}
x^2 &: a-a=0\\
x &: -2a+b+ba-3b+2b=1\\
x^0 &: 2a-2b+3b=0
\end{align*}
\\
\begin{align*}
4a=1 &\rightarrow a=\frac{1}{4}\\
2a+b=0 &\rightarrow b=-2a=-\frac{1}{2}
\end{align*}
\begin{equation*}
(y_1)_p = (\frac{1}{4}x-\frac{1}{2})e^{-x}\cdot x
\end{equation*}
\begin{align*}
y_p &= (y_1)_p +(y_2)_p\\
&= (ax^2 +bx)e^{-x} +(cx+d)e^{-2x}\cdot x
\end{align*}

\section{Method of Variation of Parameters}
$y'' +P(x)y' +Q(x)y =f(x), y_p=?$\\
\\
\underline{1 step} We solve $y'' +P(x)y' +Q(x)y =0~~~^{(2)}$ we find solutions $y_1(x), y_2(x)$
\begin{itemize}
	\item linear independent
\end{itemize}
$y =C_1y_1(x) +C_2y_2(x)$ - general $(2)$ solution of $(2)$\\
\\
\underline{2 step} We'll search for solution $y_p~~~^{(1)}$ of $(1)$ as $y_p=u_1(x)\cdot y_1(x) +y_2(x)y_2(x)$
\begin{align*}
y_p' &= u_1'(x) \cdot y_1(x) +u_1(x) \cdot y_1'(x) +u_2'(x)y_2(x) +u_2(x) \cdot y_2'(x)\\
y_p' &= u_1(x)y_1'(x) +u_2(x)y_2'(x) +u_1'y_1(x) +u_2'(x)y_2(x)
\end{align*}
Add restrictions\\
$u_1'(x) y_1(x) + u_2'(x)y_2(x)=0$\\
\begin{align*}
y_p' &= u_1(x)y'(x) +u_2(x)y_2'(x)\\
(y_p)'' &= u_1'y_1' +u_1y_1'' +u_2'y_2' +u_2 \cdot y_2''\\
\\
y_p' &= u_1(x)y_1'(x) +u_2(x)y_2'(x)
\end{align*}
from (1) we have:
\begin{align*}
f(x) &= u_1'y_1' +u_1 \cdot y_2'' +u_2'y_2' +u_2y_2'' +P(x)(u_1y_1' +u_2y_2') +Q(x)(u_1y_1 +u_2y_2)\\
y_p'' &= u_1'y_1' +u_1 \cdot y_2'' +u_2'y_2' +u_2y_2''\\
y_p' &= u_1y_1' +u_2y_2'\\
y_p &= u_1y_1 +u_2y_2
\end{align*}
\begin{align*}
u_1[y_1'' +py_1' +Qy1] +u_2[y_2'' +P \cdot y_2' +Qy_2] +u_1'y_1' +u_2'y+2' &= f(x)\\
y_1'' +py_1' +Qy1 &= 0\\
y_2'' +P \cdot y_2' +Qy_2 &= 0
\end{align*}
\begin{align*}
(3) \rightarrow u_1' &= \int \limits \phi(x)dxy_1 +\int \limits \psi(x)dx \cdot y_2(x)\\
&\nwarrow \text{parameter solution of }(1)
\end{align*}
\\
\textbf{Example 1}\\
$y'' +y =\tan x$
\begin{align*}
\text{\underline{1 step}} ~~~ &y'' +y =0 \leftarrow y=e^{rx}\\
\rightarrow &r^2 +1 =0, r_1,\\
\rightarrow &y_1 =\cos x, y_2=\sin x\\
&y = C_1 \cos x +C_2 \sin x
\end{align*}
\underline{2 step} $~~~ y_p =u_1(x)\cdot \cos x +u_2(x) \sin x$ system $(3)$ in this case is, following
\begin{equation*}
\begin{cases}
	u_1'\cos x +u_2'\sin x =0\\
	u_1'(\sin x) +u_2'(\cos x) =\tan x
\end{cases}
\Bigm|
\begin{matrix}
\sin x\\
\cos x
\end{matrix}
(+)
\end{equation*}

\begin{equation*}
0 +u_2'(\sin^2x +\cos^2x) =\tan x \cos x
\end{equation*}

\begin{align*}
u_2' &= \sin x\\
u_1' &= -u_2' \cdot \frac{\sin x}{\cos x}\\
&= -\frac{\sin x \sin x}{\cos x}\\
\\
u_2 &= \int \limits u_2' dx\\
&= \int \limits \sin x dx\\
&= -\cos x\\
\\
u_1 &= \int \limits u_1' dx\\
&= \int \limits \frac{\sin^2x}{\cos x} dx\\
&= -\frac{\sin^2x}{\cos^2x} \cos x dx\\
dv &= \cos x dx\\
\\
v &= \sin x, dv = \cos xdx\\
&= -\int \limits \frac{v^2 dv}{1-v^2}\\
&= -\int \limits \frac{(v^2-1+1)}{v^2-1}dv\\
&= \int \limits dv +\int \limits \frac{dv}{v^2-1}\\
&~~~(v-1)(v+1) = v^2-1\\
&= \int \limits dv +\int \limits \frac{dv}{(v-1)(v+1)}\\
&= v +\int \limits \frac{1}{2} \Bigm[ \frac{1}{v-1} -\frac{1}{v+1} \Bigm]dv\\
\\
u_1 &= v +\frac{1}{2} [\ln|v-1| -\ln|v+1|]\\
&= \sin x +\frac{1}{2} \ln \Bigm|\frac{\sin x-1}{\sin x+1}\Bigm|\\
\\
y_p &= u_1y_1 +u_2y_2\\
&= \Bigm(\sin +\frac{1}{2}\ln \Bigm|\frac{\sin x-1}{\sin x+1}\Bigm| \Bigm) \cos x -\cos x \sin x
\end{align*}

\section{Variation of Parameters}
\begin{align*}
y'' +P(x)y' +Q(x)y &= f(x)\\
y_p &= ?
\end{align*}

\underline{Step 1}
\begin{align*}
y'' +P(x)y' +Q(x) &= 0, y_1(x), y_2(x)\\
y_c &= C_1y_1(x) +C_2y_2(x)
\end{align*}

\underline{Step 2}
\begin{align*}
y_p &= u_1(x)y_1(x) +u_2(x)y_2\\
&\begin{Bmatrix}
	u_1'(x) \cdot y_1(x) +u_2'(x) \cdot y_2(x) =0\\
	u_1'(x)y_1'(x) +u_2'(x)y_2'(x) =f(x)
\end{Bmatrix}\\
u_1'(x) &= \rho(x), u_2'(x)=\psi(x)
%\rho is probably \phi, but \rho matches the character better
%use a different font for \phi to get the right character
\end{align*}

\textbf{Problem 49}\\
$y''-4y'+4y=2e^{2x}$\\
\underline{Step 1}
\begin{align*}
y''-4y'+4y &= 0\\
y &= e^{rx}\\
r^2-4r+4 &= 0\\
(r-2)^2 &= 0\\
r_1=2, & r_2=2\\
&\rightarrow y_1=e^{2x}, y_2=e^{2x}\cdot x
\end{align*}
\underline{Step 2}
\begin{align*}
y_p &= u_1(x) \cdot y_1(x) +u_2(x)y_2\\
&= u_1(x) \cdot e^{2x} +u_2(x)e^{2x}x\\
&\begin{cases}
	u_1'e^{2x} +u_2'e^{2x} =0\\
	u_1' \cdot 2e^{2x} + u_2'(x)(2e^{2x}x+e^{2x})=2e^{2x}
\end{cases}
\Bigm| \frac{1}{e^{2x}}\\
&\begin{cases}
	u_1'+u_2'x=0\\
	2u_1'+u_2'(2x+1)=0
\end{cases}\\
0+u_2'(2x-2x-1) &= -2\\
\\
u_1' &= 2\\
\\
u_1' &= -2\\
u &= \int \limits u_1'(x)dx\\
&= -2 \int \limits x dx\\
&= \frac{-2x}{2}\\
\\
u_2 &= \int \limits u_2'(x)dx\\
&= \int \limits 2 dx\\
&= 2x\\
y_p &= -x^2e^{2x} +2xe^{2x}x\\
&= x^2e^{2x}
\end{align*}

\section{Linear Systems of First Order Differential Equations}
\begin{align*}
&y_1(t), y_2(t), \cdots, y_n(t) - \text{unknown}\\
\frac{dy_1}{dt} &= p_{11}y_1 +p_{12}y_2 +\cdots +p_{1n}y_n +f_1(t)\\
\frac{dy_2}{dt} &= p_{21}y_1 +p_{22}y_2 +\cdots +p_{2n}y_n +f_2(t)\\
\cdots & \cdots \cdots \cdots \cdots \cdots \cdots \cdots \cdots \cdots \cdots \cdots \cdots\\
\frac{dy_n}{dt} &= p_{n1}y_1 +p_{n2}y_2 +\cdots +p_{nn}y_n +f_n(t)
\end{align*}

\noindent$p_{2i}(t)$ -given\\
$f_i(t)$ -given

\begin{equation*}
Y'(t) = 
\begin{bmatrix}
	y_1'\\
	y_2'\\
	\vdots\\
	y_n'
\end{bmatrix},
Y= \begin{bmatrix}
	y_1\\
	y_2\\
	\vdots\\
	y_n'
\end{bmatrix},
A = \begin{bmatrix}
	p_{11} & p_{12} & \cdots & y_{1n}\\
	p_{21} & p_{22} & \cdots & y_{2n}\\
	\cdots & \cdots & \cdots & \cdots\\
	p_{1n} & p_{2n} & \cdots & y_{nn}
\end{bmatrix},
F = \begin{bmatrix}
	f_1\\
	f_2\\
	\vdots\\
	f_n
\end{bmatrix}
\end{equation*}

\noindent$y(0)=y_{10}, y_2(0)=y_{20}, \cdots, y_n(0)=y_{n0}$\\
Initial Conditions

\section{Systems of 2 Equations for $x(t), y(t)$}
\begin{align*}
\frac{dx}{dt} &= a_{11}x +a_{12}y +f(t)\\
\frac{dy}{dt} &= a_{21}x +a_{22}y +f(t)
\end{align*}

\begin{equation*}
Y=\begin{bmatrix}
	x(t)\\
	y(t)
\end{bmatrix},
Y'=\begin{bmatrix}
	x'(t)\\
	y'(t)
\end{bmatrix},
A=\begin{bmatrix}
	a_{11} & a_{12}\\
	a_{21} & a_{22}
\end{bmatrix},
F=\begin{bmatrix}
	f_1\\
	f_2
\end{bmatrix}
\end{equation*}

\begin{align*}
\begin{bmatrix}
	x'\\
	y'
\end{bmatrix}
&= \begin{bmatrix}
	a_{11} & a_{12}\\
	a_{21} & a_{22}
\end{bmatrix}
\cdot \begin{bmatrix}
	x\\
	y
\end{bmatrix}
+ \begin{bmatrix}
	f_1\\
	f_2
\end{bmatrix}\\
&= \begin{bmatrix}
	a_{11}x+a_{12}y\\
	a_{12}x+a_{22}y
\end{bmatrix}
+ \begin{bmatrix}
	f_1\\
	f_2
\end{bmatrix}\\
&= \begin{bmatrix}
	a_{11}x+a_{12}y+f_1\\
	a_{12}x+a_{22}y+f_2
\end{bmatrix}
\end{align*}

\begin{align*}
x' &= a_{11}+a_{12}y+f_1\\
y' &= a_{22}+a_{22}y+f_2
\end{align*}

\noindent\textbf{Example}\\
$ay''+by'+cy=t^2$\\
$\rightarrow$ second order differential equation\\
$y'=x$\\
\\
\begin{align*}
ax'+bx+cy=t^2\\
x'=\frac{-bx-cy+t^2}{a}\\
\begin{cases}
	x'=-\frac{bx}{a}-\frac{c}{a}y+\frac{t^2}{a} & x(t)\\
	y'=x
\end{cases}
\end{align*}

\section{Method of Elimination}
\textbf{Example 1}\\
\begin{align*}
x' &= -2xy, y'=\frac{1}{2}x\\
&\downarrow\\
(x'(t)) &= (-2y)' \leftarrow \text{derivative with respect to $t$}\\
x''(t) &= -2y'(t)\\
y'(t) &= -\frac{x''(t)}{2}\\
\\
-\frac{x''(t)}{2} &= \frac{1}{2}x\\
\rightarrow x''(t) + x &= 0 \leftarrow xe^{rt}\\
r^2 + 1 &= 0, r_{1, 2} = \pm 2\\
x_1 &= \cos t, x_2 = \sin t\\
\\
x &= C_1x_1 + C_2x_2\\
&= C_1\cos t + C_2\sin t\\
y &= -\frac{x'(t)}{2}\\
&= -\frac{1}{2}(-C_1\sin t +C_2\cos t)\\
&= \frac{C_1}{2}\sin t +\frac{C_2}{2}\cos t
\end{align*}

Using Initial Conditions
\begin{align*}
x(0) &= C_1\cos0 +C_2\sin0 = 0\\
y(0) &= \frac{C_1}{2}\sin0 -\frac{C_2}{2}\cos0 = 0\\
\\
C_1 &= 2\\
C_2 &= 0\\
\\
x &= 2\cos t\\
y &= \frac{2}{2} \sin t\\
\\
\Bigm( \frac{x}{2} \Bigm)^2 &= (\cos t)^2\\
\underline{(y)^2} &\underline{= (\sin t)^2}\\
\Bigm( \frac{x}{2} \Bigm)^2 +y^2 &= \cos^2t +\sin^2t =1
\end{align*}

\begin{align*}
\frac{y^2}{2^2} + \frac{y^2}{2^2} = 1
%KIR: Include graphic of the graph of the function
\end{align*}

\textbf{Example 2}
\begin{align*}
(1) x'(t) &= 4x-3y\\
(2) y'(t) &= 6x-7y\\
\\
\text{from }(2) x &= \frac{y'(t)+7y}{6}\\
x'(t) &= \frac{y''+7y'}{6}\\
\text{from }(1) \frac{y''+7y'}{6} &= 4\frac{y'+7y}{6} -3y\\
\\
y''+7y' &= 4y'+28y-18y\\
y''+3y &= 0 \leftarrow y=e^{rt}\\
r^2+3r-10 &= 0\\
r_1 &= 2, r_2=-5\\
\\
y_1 &= e^{2t}\\
y_2 &= r^{-5t}\\
\\
y &= C_1e^{2t} +C_2e^{-5t}\\
\\
x &= \frac{2C_1e^{2t}-5C_2e^{-5t}+7(C_1e^{2t}+C_2e^{-5t})}{6}\\
&= \frac{9C_1e^{2t}}{6} +\frac{2C_2e^{-5t}}{6}\\
&= x\\
\\
y &= C_1e^{2t} +C_2e^{-5t}
\end{align*}

\section{System of first order Differential Equations}
\begin{align*}
\overline{X}' &= \overline{A} \cdot \overline{X} \rightarrow \begin{bmatrix}
	x'(t)\\
	y'(t)
\end{bmatrix} = \begin{bmatrix}
	a_{11} & a_{12}\\
	a_{21} & a_{22}
\end{bmatrix} \cdot \begin{bmatrix}
	x\\
	y
\end{bmatrix}\\
&\text{solutions}\\
&\begin{cases}
	x'=a_{11}x+a_{12}y\\
	y'=a_{21}x+a_{22}y
\end{cases}\\
&\begin{cases}
	x'=C_1x_1 +C_2x_2\\
	y'=C_1y_1 +C_2y_2
\end{cases}\\
\overline{X} &= C_1\overline{X_1}+C_2\overline{X_2}\\
&\text{where }\overline{X} = \begin{bmatrix}
	x\\
	y
\end{bmatrix}, \overline{X_1} = \begin{bmatrix}
	x_1\\
	y_1
\end{bmatrix}, \overline{X_2} = \begin{bmatrix}
	x_2\\
	y_2
\end{bmatrix}
\end{align*}
$\overline{X_1}, \overline{X_2}$ particular solution of system 1\\
$\overline{X_1}, \overline{X_2}$ - linear independent or $C_1\overline{X_1} + C_2\overline{X_2} = 0$, only if $C_1=C_2=0$\\
\begin{equation*}
W = \begin{vmatrix}
	x_1 & x_2\\
	y_1 & y_2
\end{vmatrix} \neq 0
\end{equation*}
\textbf{Example 1}
\begin{align*}
\overline{X'} &= \begin{bmatrix}
	4 & -3\\
	6 & -7
\end{bmatrix}
\overline{X}(t)\\
\overline{X_1} &= \begin{bmatrix}
	3e^{2t}\\
	2e^{2t}
\end{bmatrix}, \overline{X_2}\begin{bmatrix}
	e^{-5t}\\
	3e^{-5t}
\end{bmatrix}\\
\overline{X_1}, \overline{X_2} &- \text{linear independent}
\end{align*}
Substitute $\overline{X_1}$, into the system gives
\begin{align*}
\begin{bmatrix}
	(3e^{2t})'\\
	(2e^{2t})'
\end{bmatrix} &= \begin{bmatrix}
	4 & -3\\
	6 & -7
\end{bmatrix} \begin{bmatrix}
	3e^{2t}\\
	2e^{2t}
\end{bmatrix}\\
\begin{bmatrix}
	6e^{2t}\\
	4e^{2t}
\end{bmatrix} &= \begin{bmatrix}
	4\cdot 3\cdot e^{2t}+(-3)\cdot 3\cdot e^{2t}\\
	6\cdot 2\cdot e^{2t}+(-7)\cdot 2\cdot e^{2t}
\end{bmatrix}\\
&= \begin{bmatrix}
	6e^{2t}\\
	4e^{2t}
\end{bmatrix}
\end{align*}
Substitute $\overline{X_2}$ into the system:
\begin{align*}
\begin{bmatrix}
	(e^{-5t})'\\
	(3e^{-5t})'
\end{bmatrix} &= \begin{bmatrix}
	4 & -3\\
	6 & -7
\end{bmatrix} \begin{bmatrix}
	e^{-5t}\\
	3e^{-5t}
\end{bmatrix}\\
\begin{bmatrix}
	-5e^{-5t}\\
	-15e^{-5t}
\end{bmatrix} &= \begin{bmatrix}
	4e^{-5t}+(-3)3e^{-5t}\\
	6e^{-5t}+(-7)3e^{-5t}
\end{bmatrix}\\
&= \begin{bmatrix}
	-5e^{-5t}\\
	-15e^{-5t}
\end{bmatrix}
\end{align*}

\begin{align*}
W &= \begin{vmatrix}
	x_1 & x_2\\
	y_1 & y_2
\end{vmatrix} = \begin{vmatrix}
	3e^{2t} & e^{-5t}\\
	2e^{2t} & 3e^{-5t}
\end{vmatrix}\\
&= 3e^{2t} \cdot 3e^{-5t} -2e^{2t} \cdot e^{-5t}\\
&= 9e^{-3t}-2e^{-3t}\\
&= 7e^{-3t} \neq 0
\end{align*}
Linear Independent\\
\begin{align*}
\overline{X} &= C_1\begin{bmatrix}
	3e^{2t}\\
	2e^{2t}
\end{bmatrix} + \begin{bmatrix}
	C_2e^{-5t}\\
	3e^{-5t}
\end{bmatrix}\\
x &= 3C_1e^{2t}+C_2e^{-5t}\\
y &= 2C_1e^{2t}+3C_2e^{-5t}
\end{align*}

\section{Method of Eigenvalues}
$y''+ay'+b=0, y=e^{rt}$\\
\begin{equation*}
\overline{X'} = \overline{A} \cdot \overline{X} \leftarrow \overline{X} = \overline{V} \cdot e^{\lambda t}, \overline{V} = \begin{bmatrix}
	a\\
	b
\end{bmatrix}
\end{equation*}

\begin{align*}
\overline{V} \cdot \lambda e^{\lambda t} &= \overline{A} \cdot \overline{V}\\
\rightarrow \overline{A} \overline{V} &- \lambda \overline{V} = 0\\
\overline{V} &= \overline{I} \overline{V}, I=\begin{bmatrix}
	1 & 0\\
	0 & 1
\end{bmatrix}\\
&= \begin{bmatrix}
	1 & 0\\
	0 & 1
\end{bmatrix} \begin{bmatrix}
	a\\
	b
\end{bmatrix} = \begin{bmatrix}
	a\\
	b
\end{bmatrix}\\
\overline{A} \cdot \overline{V} &- \lambda I \overline{V} = 0 \leftarrow \text{ (cannot subtract constant from matrix)}\\
(\overline{A} &- \lambda I ) \cdot \overline{V} = 0\\
| (\overline{A} &- \lambda I) | = 0 \leftarrow \text{ (infintely many solutions)}\\
&\rightarrow \Bigm| \begin{bmatrix}
	a_{11} & a_{12}\\
	a_{21} & a_{22}
\end{bmatrix} - \begin{bmatrix}
	\lambda & 0\\
	0 & \lambda
\end{bmatrix} \Bigm| = 0\\
&\rightarrow \Bigm| \begin{bmatrix}
	a_{11}-\lambda & a_{12}\\
	a_{21} & a_{22}-\lambda
\end{bmatrix} \Bigm| = 0\\
&\rightarrow \Bigm| \begin{bmatrix}
	a_{11}-\lambda & a_{12}\\
	a_{12} & a_{22}-\lambda
\end{bmatrix} \Bigm| = 0\\
&(a_{11}-\lambda)(a_{22}-\lambda)-a_{12}a_{21}=0\\
&\lambda^2-\lambda(a_{11} + a_{22})+a_{11}a_{12}-a_{12}a_{22}=0
\end{align*}

(2) Two distinct roots of characteristic equations $\lambda=\lambda_1$, $\lambda=\lambda_2$, $\lambda_1 \neq \lambda_2$ substitute $\lambda$ in $(\overline{A} \rightarrow I)\overline{V}=0$
\begin{align*}
\Bigm( \begin{bmatrix}
	a_{11} & a_{12}\\
	a_{21} & a_{22}
\end{bmatrix} - \begin{bmatrix}
	\lambda_1 & 0\\
	0 & \lambda_2
\end{bmatrix} \Bigm) \begin{bmatrix}
	a\\
	b
\end{bmatrix} = 0\\
\begin{bmatrix}
	a_{11}-\lambda & a_{12}\\
	a_{21} & a_{22}-\lambda
\end{bmatrix} \begin{bmatrix}
	a\\
	b
\end{bmatrix} = 0\\
\begin{cases}
	(a_{11}-\lambda)+a_{12}b=0\\
	a_{21}a+(a_{22}-\lambda_1)b=0
\end{cases}
\end{align*}

\textbf{Problem 13}
\begin{align*}
\begin{cases}
	x' = 2x+4y+3 \cdot e^t\\
	y' = 5x-y-t^2
\end{cases}\\
\overline{X} = \begin{bmatrix}
	x\\
	y
\end{bmatrix}, \overline{F} \begin{bmatrix}
	3e^{t}\\
	-t
\end{bmatrix}\\
\begin{bmatrix}
	x'\\
	y'
\end{bmatrix} = \begin{bmatrix}
	2 & 4\\
	5 & -1
\end{bmatrix} \begin{bmatrix}
	x\\
	y
\end{bmatrix} + \begin{bmatrix}
	3e^t\\
	-t^2
\end{bmatrix}\\
\overline{X'} = \begin{bmatrix}
	2 & 4\\
	5 & -1
\end{bmatrix} \cdot \overline{X} + \overline{F}\\
\begin{bmatrix}
	x'\\
	y'
\end{bmatrix} = \begin{bmatrix}
	2x+4y\\
	5x-y
\end{bmatrix} + \begin{bmatrix}
	3e^t\\
	-t^2
\end{bmatrix}\\
= \begin{bmatrix}
	2x+4y+3e^t\\
	5x-y-t^2
\end{bmatrix}\\
\begin{cases}
	x'=2x+4y+3e^t\\
	y'=5x-y-t^2
\end{cases}
\end{align*}

\section{Method of Eigenvalues}
\begin{align*}
\overline{X'} = \overline{A} \cdot \overline{X} &\leftarrow \overline{X} = \overline{V}e^{\lambda t}, \overline{V} = \begin{bmatrix}
	a\\
	b
\end{bmatrix} \leftarrow \text{constant vector}\\
\overline{V}(e^{\lambda t})' &= \overline{A} \cdot \overline{V}e^{\lambda t}\\
\lambda \overline{V} \cancel{e^{\lambda t}} &= \overline{A} \cdot \overline{V} \cancel{e^{\lambda t}}\\
\rightarrow \overline{A} \overline{V} &- \lambda \overline{V} = 0\\
\rightarrow \overline{A} \overline{V} &- \lambda \overline{I} \overline{V} = 0
\end{align*}
\begin{align*}
\overline{I} = \begin{bmatrix}
	1 & 0\\
	0 & 1
\end{bmatrix}, \begin{bmatrix}
	1 & 0\\
	0 & 1
\end{bmatrix} \begin{bmatrix}
	a\\
	b
\end{bmatrix} = \begin{bmatrix}
	a\\
	b
\end{bmatrix}\\
\rightarrow (\overline{A} - \lambda \overline{I}) -\overline{V} = \overline{0}\\
\overline{0} = \begin{bmatrix}
	0\\
	0
\end{bmatrix}, A= \begin{bmatrix}
	a_{11} & a_{12}\\
	a_{21} & a_{22}
\end{bmatrix}\\
\rightarrow \Bigm( \begin{bmatrix}
	a_{11} & a_{12}\\
	a_{21} & a_{22}
\end{bmatrix} -\lambda \begin{bmatrix}
	1 & 0\\
	0 & 1
\end{bmatrix} \Bigm) \begin{bmatrix}
	a\\
	b
\end{bmatrix} = 0\\
\begin{bmatrix}
	a_{11} - \lambda & a_{12}\\
	a_{21} & a_{22} - \lambda
\end{bmatrix} \begin{bmatrix}
	a\\
	b
\end{bmatrix} = \begin{bmatrix}
	0\\
	0
\end{bmatrix}\\
\rightarrow \begin{bmatrix}
	(a_{11}-\lambda)a+a_{12}b\\
	a_{21}a+(a_{22}-\lambda)b
\end{bmatrix} = \begin{bmatrix}
	0\\
	0
\end{bmatrix}\\
\begin{cases}
	(a_{11}-\lambda)a+a_{12}b=0\\
	a_{21}a+(a_{22}-\lambda)b=0
\end{cases} (a,b)?\\
\begin{vmatrix}
	a_{11}-\lambda & a_{12}\\
	a_{21} & a_{22}-\lambda
\end{vmatrix} = 0\\
(a_{11} - \lambda)(a_{22} - \lambda) - a_{21}a_{12} = 0\\
\lambda^2 - (a_{11}+a_{22})\lambda + a_{11}a_{22} - a_{21}a_{12} = 0
\end{align*}
\begin{align*}
	(1) & \lambda=\lambda_1, \lambda=\lambda_2, \lambda_1 \neq \lambda_2\\
	(2) & \lambda_1 = \lambda_2\\
	(3) & \lambda_1, \lambda_2 - \text{ complex}
\end{align*}
\begin{equation*}
\overline{X_1} = \overline{V_1} e^{\lambda t}
\end{equation*}
\textbf{Problem 3}
\begin{equation*}
\begin{cases}
	x_1'=3x_1+4x_2 & x_1(0)=1\\
	x_2'=3x_1+2x_2 & x_2(0)=1
\end{cases}
\end{equation*}
\begin{align*}
x_1(t), x_2(t) ?\\
\overline{X} = \begin{bmatrix}
	x_1\\
	x_2
\end{bmatrix}, A = \begin{bmatrix}
	3 & 4\\
	3 & 2
\end{bmatrix}\\
\rightarrow \overline{X'} = \begin{bmatrix}
	3 & 4\\
	3 & 2
\end{bmatrix} \overline{X}
\end{align*}
substitute $\overline{X} = \begin{bmatrix}
	a\\
	b
\end{bmatrix} e^{\lambda t}$ after determining $\lambda$
\begin{align*}
\begin{vmatrix}
	a_{11}-\lambda & a_{12}\\
	a_{21} & a_{22}-\lambda
\end{vmatrix} &= 0\\
\begin{vmatrix}
	3-\lambda & 4\\
	3 & 2-\lambda
\end{vmatrix} &= 0\\
(3-\lambda)(2-\lambda)-3-4 &= 0\\
\lambda^2-5\lambda-6 &= 0\\
\lambda = \lambda_1 = 6, ~~~~~ \lambda &= \lambda_2 = -1\\
\\
\begin{bmatrix}
	a_{11}-\lambda & a_{12}\\
	a_{21} & a_{22}-\lambda
\end{bmatrix} \begin{bmatrix}
	a\\
	b
\end{bmatrix} &= 0\\
\begin{cases}
	(a_{11}-\lambda)a+a_{12}b=0\\
	a_{21}a+(a_{22}-\lambda)b=0
\end{cases}&\\
\downarrow ~~~~~~~~~~ &\\
\begin{cases}
	(3-\lambda)a+4b=0\\
	3a+(2-\lambda)b=0
\end{cases}& \\
\end{align*}
(2) $\lambda = \lambda_1=b$\\
\begin{align*}
\begin{cases}
	(3-6)a+4b=0\\
	3a+(2-6)b=0
\end{cases}& \\
\begin{cases}
	-3a+4b=0\\
	3a-4b=0
\end{cases}& \\
\end{align*}
Test for valid equations, expect unlimited solutions\\
Chose any value for $a$ and $b$\\
\begin{align*}
a= 4, b &= 3 \\
\rightarrow \begin{bmatrix}
	4\\
	3
\end{bmatrix} \rightarrow \overline{X_1} &= \begin{bmatrix}
	4\\
	3
\end{bmatrix} e^{\lambda_1 t}\\
&= \begin{bmatrix}
	4\\
	3
\end{bmatrix}
	e^{6t}
\end{align*}
(3) $\lambda =\lambda_2 =-1$\\
\begin{align*}
&\begin{cases}
	(3-(-1))a+4b=0\\
	3a+(3-(1))b=0
\end{cases}\\
&\begin{cases}
	4a+4b=0\\
	3a+3b=0
\end{cases}\\
&\text{Test for valid equation}
\end{align*}
choose values for $a$ and $b$\\
\begin{align*}
a &= -1, a=1\\
\overline{X_2} &= \begin{bmatrix}
	1\\
	-1
\end{bmatrix} e^{\lambda_2t} = \begin{bmatrix}
	1\\
	-1
\end{bmatrix} e^{-t}\\
\begin{bmatrix}
	x_1\\
	x_2
\end{bmatrix} &= C_1 \begin{bmatrix}
	4\\
	3
\end{bmatrix} e^{6t} + C_2 \begin{bmatrix}
	1\\
	-1
\end{bmatrix} e^{-t}\\
&= \begin{bmatrix}
	4C_1e^{6t}\\
	3C_1e^{6t}
\end{bmatrix} + \begin{bmatrix}
	C_2e^{-t}\\
	-C_2e^{-t}
\end{bmatrix}\\
&= \begin{bmatrix}
	4C_1e^{6t}+C_2e^{-t}\\
	3C_1e^{6t}-C_2e^{-t}
\end{bmatrix}\\
& \begin{cases}
	x_1 = 4C_1e^{6t}+C_2e^{-t}\\
	x_2 = 3C_1e^{6t}-C_2e^{-t}
\end{cases}
\end{align*}
Using Initial Conditions
\begin{align*}
\begin{matrix}
	x_1(0)=4C_1e^0+C_2e^0=1\\
	x_2(0)=3C_1e^0-C_2e^0=1
\end{matrix} \Bigm\} \\
\\
\begin{cases}
	4C_1+C_2=1\\
	3C_1-C_2=1
\end{cases} \oplus \\
\\
7C_1+0 &= 2\\
C_1 &= \frac{2}{7}\\
\\
4 \Bigm( \frac{2}{7} \Bigm) + C_2 &= 1\\
C_2 &= 1-4 \cdot \frac{2}{7}\\
&= -\frac{1}{7}\\
\\
x_1 &= \frac{8}{7}e^{6t}+ \Bigm( -\frac{1}{7} \Bigm) e^{-t}\\
x_2 &= \frac{6}{7}e^{6t}+ \frac{1}{7}e^{-t}
\end{align*}
\section{Eigenvalue Method to $\overline{X'}$}
\begin{align*}
\overline{X'} &= \begin{bmatrix}
	a_{11} & a_{12}\\
	a_{21} & a_{22}
\end{bmatrix} \cdot \overline{X}\\
\overline{X'} &= \begin{bmatrix}
	a\\
	b
\end{bmatrix} e^{\lambda t}
\end{align*}
Characteristic Equation
\begin{align*}
\begin{vmatrix}
	a_{11}-\lambda & a_{12}\\
	a_{21} & a_{22}-\lambda
\end{vmatrix} =0\\
\rightarrow \lambda=\lambda_1, \lambda=\lambda_2
\end{align*}
System of Equations for ``$a$'' and ``$b$''
\begin{align*}
&\begin{bmatrix}
	a_{11}-\lambda & a_{12}\\
	a_{21} & a_{22}-\lambda
\end{bmatrix} \begin{bmatrix}
	a\\
	b
\end{bmatrix} = 0\\
&\rightarrow \begin{cases}
	(a_{11}-\lambda)a+a_{21}b=0\\
	a_{21}a+(a_{22}-\lambda)b=0
\end{cases}\\
&\rightarrow \begin{bmatrix}
	a\\
	b
\end{bmatrix}
\end{align*}
2. $\lambda$ - complex number\\
$\lambda_1=\alpha+i\beta, \lambda_2=\alpha-i\beta$
\begin{align*}
\begin{cases}
	(a_{11}-\lambda)a+a_{12}b=0\\
	a_{21} \cdot a+(a_{22}-\lambda_1)b=0
\end{cases}\\
\begin{cases}
	(a_{11}-\alpha-i\beta)a+a_{12}b=0\\
	a_{21}a+(a_{22}-\alpha-i\beta)b=0
\end{cases}\\
\rightarrow \begin{pmatrix}
	a=a_1+ia_2\\
	b=b_1+ib_2
\end{pmatrix}\\
\overline{X} = \begin{bmatrix}
	a_1+ia_2\\
	b_1+ib_2
\end{bmatrix} e^{(\alpha+i\beta)}t
\end{align*}
Euler's Formula
\begin{align*}
\rightarrow & e^{(\alpha+i\beta)}\\
&= e^{\alpha t}(\cos \beta t \pm i\sin \beta t)\\
\overline{X} &= \begin{bmatrix}
	a_1 + ia_2\\
	b_1 + ib_2
\end{bmatrix} e^{\alpha t}(\cos \beta t \pm i \sin \beta t)\\
&= \begin{bmatrix}
	(a_1+ia_2)(\cos \beta t + i \sin \beta t)\\
	(b_1+ib_2)(\cos \beta t + i \sin \beta t)
\end{bmatrix} e^{\alpha t}\\
&= \begin{bmatrix}
	a_1\cos \beta t + i(a_1\sin \beta t + a_2\cos \beta t) - a_2 \sin \beta t\\
	b_1\cos \beta t + i(b_1\sin \beta t + b_2\cos \beta t) + b_2 \sin \beta t
\end{bmatrix} e^{\alpha t}\\
&= \begin{bmatrix}
	e^{\alpha t} (a_1 \cos \beta t -a_2\sin \beta t)\\
	e^{\alpha t} (b_1 \cos \beta t -b_2\sin \beta t)
\end{bmatrix} +i \begin{bmatrix}
	e^{\alpha t} (a_1\sin \beta t + a_2 \cos \beta t)\\
	e^{\alpha t} (b_1\sin \beta t + b_2 \cos \beta t)
\end{bmatrix}\\
\overline{X} &= \overline{X_1} + i\overline{X_2}\\
\overline{X_1} &= \text{Re}[\overline{X}], \overline{X_2} = \text{Im}[\overline{X}]\\
& \text{- Particular Solutions}
\end{align*}
\textbf{Example 1}\\
\begin{align*}
\frac{dx_1}{dt} &= 4x_1-3x_2\\
\frac{dx_2}{dt} &= 3x_1+4x_2\\
\rightarrow A &= \begin{bmatrix}
	4 & -3\\
	3 & 4
\end{bmatrix}\\
&= \begin{bmatrix}
	a_{11} & a_{12}\\
	a_{21} & a_{22}
\end{bmatrix}
\end{align*}
Characteristic Equation
\begin{align*}
\begin{vmatrix}
	4-\lambda & -3\\
	3 & 4-\lambda
\end{vmatrix} = 0\\
(4-\lambda)(4-\lambda)-3\cdot(-3) &= 0\\
\rightarrow (\lambda-4)^2+9 &= 0\\
(\lambda-4)^2 &= -9\\
\lambda-4 &= \pm 3i\\
\lambda_1=4+3i, \lambda_2 &= 4-3i
\end{align*}
\begin{align*}
\text{for } \begin{bmatrix}
	a\\
	b
\end{bmatrix} &: \begin{bmatrix}
	4-\lambda_2 & -3\\
	3 & 4-\lambda_2
\end{bmatrix} \begin{bmatrix}
	a\\
	b
\end{bmatrix} = 0\\
&\begin{cases}
	4-(4(4-3i))\cdot a+(-3)b=0\\
	3a+(4(4-3i))b=0
\end{cases}\\
&\begin{cases}
	3ia-3b=0\\
	3a+3ib=0
\end{cases}\\
&\begin{cases}
	ia-b=0\\
	a+ib=0
\end{cases}\\
&\begin{matrix}
	\rightarrow ia-b=0\\
	ia-b=0
\end{matrix}\\
&a=i, b=-1
\end{align*}
\begin{align*}
x &= \begin{bmatrix}
	i\\
	-1
\end{bmatrix} e^{(4-3i)t}\\
&= \begin{bmatrix}
	i\\
	-1
\end{bmatrix} e^{4y}(\cos 3t-i\sin 3t)\\
&= \begin{bmatrix}
	i(\cos3t-i\sin3t)e^{4t}\\
	-1(\cos3t-i\sin3t)e^{4t}
\end{bmatrix}\\
&= \begin{bmatrix}
	(i\cos3t+\sin3t)e^{4t}\\
	(-\cos3t+i\sin3t)e^{4t}
\end{bmatrix}\\
\overline{X_1} &= \text{Re}[\overline{X}]\\
&= \begin{bmatrix}
	\sin3t \cdot e^{4t}\\
	-\cos 3t \cdot e^{4t}
\end{bmatrix}\\
\overline{X_2} &= \text{Im}[\overline{X}]\\
&= \cancel{i}\begin{bmatrix}
	\cos3t \cdot e^{4t}\\
	\sin3t \cdot e^{4t}
\end{bmatrix}\\
&= \begin{bmatrix}
	\cos3t \cdot e^{4t}\\
	\sin3t \cdot e^{4t}
\end{bmatrix}\\
\overline{X} &= C_1\overline{X_1}+C_2\overline{X_2}\\
&= C_1\begin{bmatrix}
	\sin3te^{4t}\\
	-\cos3te^{4t}
\end{bmatrix} + C_2 \begin{bmatrix}
	\cos3te^{4t}\\
	\sin3te^{4t}
\end{bmatrix}\\
&= \begin{bmatrix}
	x_1\\
	x_2
\end{bmatrix}\\
x_1 &= C_1\sin3te^{4t}+C_2\cos3te^{4t}\\
x_2 &= -C_1\cos3te^{4t}+C_2\sin3te^{4t}
\end{align*}
\end{document}