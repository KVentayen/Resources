\documentclass[12pt, letterpaper]{article}
\usepackage[letterpaper, portrait, margin=1in]{geometry}   %For page Setup
\usepackage[utf8]{inputenc}
\usepackage{amssymb, amsmath}               %For Equations and Formulas
\usepackage{comment}                        %For Commenting
\usepackage{hyperref}                       %For Hyperlinks
\usepackage{listings}                       %For Coding Examples
\usepackage[table]{xcolor}                  %For Coloring Tables
\usepackage{xcolor}                         %For Color Associated with Coding Examples
\usepackage{multicol}                       %For Making Multiple Columns
\usepackage{multirow}                       %Allows for multiple cells in one row in a table
\usepackage{graphicx, epstopdf}                       %Converts eps files to pdf
\epstopdfsetup{update}

\title{Math Notes}
\author{K}
\date{March 6, 2020}

\usepackage{natbib}
\usepackage{graphicx}

\hypersetup{                                %Setup for Hyperlink Colors
    colorlinks=true,
    linkcolor=blue,                         %For Hyperlinked Text
    filecolor=magenta,                      %For Text that Hyperlinks to other Files
    urlcolor=cyan,                          %For Hyperlinked Printed URLs
}



\begin{document}

\section{Equation Sheet 202A (Physics)}
\begin{enumerate}
  \item $v_f = v_0 + at$
  \item $x_f = x_0 + (v_0)t + \frac{1}{2} a t^2$
  \item $v_f^2 = v_0^2 + 2a (\Delta x)$
        where $\Delta x = x_f - x_0$\\
        $x_0 = \text{initial displacement}$\\
        $x_f = \text{final displacement}$\\
        $v_0 = \text{inital velocity}$\\
        $v_f = \text{final velocity}$\\
        $a = \text{acceleration}$\\
        $t = \text{time interval for initial \& final positions}$
  \item $\text{velocity} (v) = \frac{\Delta x}{\Delta t}$\\
        $\Delta x = \text{displacement change}$\\
        $\Delta t = \text{time interval}$
  \item $\text{acceleration} (a) = \frac{\Delta v}{\Delta t}$
        $\Delta v = velocity change$
        $\Delta t = time interval$
\end{enumerate}

Equation of straight line:\\
% Re: Add Straight Line Equation
$y=mx+b$\\
$b = \text{y-intercept}$\\
$m = slope = \frac{\Delta x}{\Delta y}$

\begin{enumerate}
  \item Trigonometric Cheet-Sheet\\
        % Re: Edit numbering starting here.
        % Re: Add Trigonometric Triangle
        $h = \text{hypotenuse}$\\
        $p = \text{opposite}$\\
        $b = \text{adjacent}$
        \begin{enumerate}
          \item $\sin \theta = \frac{p}{h}$
          \item $\cos \theta = \frac{b}{h}$
          \item $\tan \theta = \frac{p}{b}$
          \item $h^2 = p^2 + b^2$
          \item $\sin^2 \theta + \cos^2 \theta = 1$
          \item $180 ^{\circ} = \pi \cdot radian$
        \end{enumerate}
        \item $\text{Area of triangle} = \frac{1}{2} \cdot base \cdot height$
        % Re: Add Image of the Area of the Triangle
        \item $\text{Area of rectangle} = length \cdot width$\\
              $Perimeter = 2(length + width)$
              % Re Add Image of the Area of the Rectangle
        \item Quadratic Equation\\
              $ax^2 + bx + c = 0$\\
              $x = \frac{-b \pm \sqrt{b^2 - 4ac}}{2a}$
        \item 12 inch = 1 foot\\
              3 foot = 1 yard\\
              1 mile = 1.61 km\\
              1 km = 1000 m\\
              100 cm = 1 m\\
              10 mm = 1 cm\\
              1 hr = 3600 sec\\
              1 inch = 2.54 cm
        \item x-component of $\overrightarrow A$:\\
              $A_x = A \cos \theta$
              % Re: Add in component graph
              y-component of $\overrightarrow A$:\\
              $A_y = A \sin \theta$\\
              $A_x^2 + A_x^2 = A^2$\\
              $\tan \theta = \frac{A_y}{A_x}$, $\sin \theta = \frac{A_y}{A}$, $\cos \theta = \frac{A_x}{A}$
        \item Acceleration due to gravity (g)\\
              $g = 9.8 m/s^2 = 10 m/s^2$
        \item Weight of mass \'m\'\\
              $= mg = \text{force of Earth on mass} (\overrightarrow F_{\text{E on o}})$
        \item $\mu_{max} = \frac{f_{s, max}}{F^{\perp}}$\\
              $F^\perp = \text{normal force}$\\
              $f_{s, max} = \text{limiting static friction}$\\
              $\mu_{s, max} = \text{coefficient of limiting static friction}$
        \item $\mu_k = \frac{f_k}{F^{\perp}}$\\
              $f_k = \text{kinetic friction}$\\
              $\mu_k = \text{coefficient of kinetic friction}$
        \item If the net force acting on an object is zero,\\
              $\sum F_x = 0$ (net force along x-axis)\\
              $\sum F_y = 0$ (net force along y-axis)
        \item Newton's $2^{nd}$ law of motion\\
              $a = \frac{\sum F}{m}$\\
              $m = \text{mass}$\\
              $a = \text{acceleration}$\\
              $\sum F =\text{net force}$
        \item For a projectile motion:\\
              $a_x = 0 \text{(acceleration along the x-axis)}$\\
              $a_y = -g \text{(acceleration along y-axis)}$\\
              If `upward' is considered as `positive' and `downward' is considered as `negative'.
        \item Newton's $3^{rd}$ law of Motion\\
              % Re: Insert diagram demonstrating Newton's 3rd Law
              $\overrightarrow F_{1 2} = \text{force exerted by object 2 on object 1}$\\
              $\overrightarrow F_{2 1} = \text{force exerted by object 1 on object 2}$
        \item Circular Motion
        \begin{enumerate}
          \item $\text{Angular Velocity} (\omega) = \frac{\Delta \theta}{\Delta t} = \frac{3\pi}{T} \text{rad/s}$\\
          (T = time period)
          \item $\text{frequency (f)} = \frac{1}{T}$
          \item $v = r\omega$\\
          $r$ = radius\\
          $v$ = velocity
          \item $\text{circumference of a circle} = 2\pi r$
          \item centripetal acceleration $(a_c) = \frac{v^2}{r}$
          \item centripetal force $(F_c) = ma_c = \frac{mv^2}{r}$
        \end{enumerate}
        \item Static Equilibrium
        \begin{enumerate}
          \item $\sum F = 0 \Rightarrow \sum F_x = 0 \text{(net force along x-axis)}$\\
          $\sum F_y = 0 \text{(net force along the y-axis)}$
          \item $\text{Torque} = F d \sin \theta$\\
          $F = \text{force}$\\
          $d = \text{distance}$\\
          $\theta = \text{angle between $\overrightarrow F$ \& $\overrightarrow d$}$
        \end{enumerate}
        \item
        \begin{enumerate}
          \item $\text{linear momentum} (\overrightarrow p) = mass \cdot velocity$\\
          $= m \overrightarrow v$
          \item $\text{Impulse} = force \cdot time interval$\\
          $= \overrightarrow F (\Delta t)$\\
          $= m \overrightarrow v_f - m \overrightarrow v_0$\\
          $\overrightarrow v_f = \text{final velocity}$\\
          $\overrightarrow v_0 = \text{initial velocity}$
          \item conservation of linear momentum\\
          $m_1 v_{0, 1} + m_2 v_{0, 2} = m_1 v_{f, 1} + m_2 v_{f, 2}$\\
          $v_{0, 1} = \text{Inital Velocity of mass $m_1$}$\\
          $v_{0, 2} = \text{Initial velocity of mass $m_2$}$\\
          $v_{f, 1} = \text{Final velocity of mass $m_1$}$\\
          $v_{f, 2} = \text{Final Velocity of mass $v_2$}$\\
          $\sum_{initial} m v = \sum_{final} m v$
          \item Momentum-impulse relation
          $I = p_f - p_0$\\
          $p_f = \text{final momentum}$\\
          $p_0 = \text{Initial momentum}$
        \end{enumerate}
        \item Work and Energy
        \begin{enumerate}
          \item $\text{Work (w)} = F d \cos \theta$\\
          $F = \text{force}$\\
          $d = \text{displacement}$\\
          $\theta = \text{angle between $\overrightarrow F \& \overrightarrow d$}$
          \item $\text{Kinetic Energy (K)} = \frac{1}{2} m v^2 = \frac{p^2}{2m}$\\
          $m = \text{mass}$\\
          $v = \text{velocity}$\\
          $p = mv = \text{momentum}$
          \item $\text{Gravitational potential energy} (U_g) = m g h$\\
          $m = \text{mass}$\\
          $g = \text{acceleration due to gravity}$\\
          $h = \text{height from a reference level}$
          \item $\text{power (P)} = \frac{Work}{time} = \frac{W}{t}$\\
          \item $\text{Elastic potential energy} (v_e) = \frac{1}{2} k x^2$\\
          $k = \text{spring constant (or force constant)}$\\
          $x = \text{extension (i.e., change in length)}$
          \item Generalised work-energy principle\\
          $K_0 + v_0 + W = K_f + v_f + \Delta v_{int}$\\
          $K_0 = \text{Initial kinetic energy}$\\
          $v_0 = \text{Initial potential energy}$\\
          $K_f = \text{Final kinetic energy}$\\
          $v_f = \text{Final potential energy}$\\
          $W = \text{Work done by external force}$\\
          $\Delta v_{int} = \text{change in internal energy}$
          \item Hooke's law\\
          $\overrightarrow {F_s} = -K \overrightarrow x$\\
          $\overrightarrow {F_s} = \text{force due to spring}$\\
          $\overrightarrow x = \text{extension or change in length}$\\
          $K = \text{Spring constant or force constant}$
        \end{enumerate}
        \item Total energy of a vibrating body ($v$)\\
          $v = \frac{1}{2} m v^2 + \frac{1}{2} K x^2$\\
          % Diagram of a Vibrating Body
          ($m = \text{mass}$, $v = \text{velocity when displacement is x from equibrium position}$, $K= \text{Spring constant}$)\\
        \item For simple pendulum,\\
          $T = 2 \pi \sqrt{\frac{l}{g}}$\\
          ($T = \text{time period}$, $l = \text{length of pendulum}$, $g = 9.81 m/s^2$)
        \item $\text{Angular velocity ($\omega$)} = \frac{\Delta \theta}{\Delta t} = \frac{2 \pi}{T} = 2 \pi f$\\
          ($T = \text{Time period}, \Delta \theta = \text{angle in angle}, \Delta t = \text{time interval}, f = \text{frequency} = \frac{1}{T}$)
        \item Simple Harmonic motion:\\
          $x = A \cos \omega t$\\
          $= A cos (\frac{2 \pi}{T}t)$\\
          ($x = \text{displacement from equilibrium position}, A = \text{amplitude}, \omega = \text{angular velocity}, T = \text{time period}$)\\
          \\
          $v = -v_{max} \sin \omega t$\\
          $v = -\omega A \sin \omega t$\\
          ($v = \text{velocity at a time $t$}, v_{max} = \omega A = \text{maximum velocity}$)\\
          \\
          $a = -a_{max} \cos \omega t$\\
          $a = \omega^2 A \cos \omega t$\\
          $a = -\omega^2 x$\\
          ($a_{max} = \omega^2 A = \text{maximum acceleration}, a = \text{acceleration at time t}$)
        \item $T = 2 \pi \sqrt{\frac{m}{K}}$\\
          ($T = \text{time period for horizontal/vertical spring-mass system}, m = \text{mass attached to the spring}, K = \text{spring constant}$)
        \item Wave motion: $v: f \lambda$\\
          ($v = \text{velocity}, f = \text{frequency}, \lambda = \text{wavelength}$)\\
          $f = \frac{1}{T}$ ($T = \text{time period}$)
        \item waves in string fixed at both ends:\\
          $v = \sqrt{\frac{T}{\mu}}$\\
          ($v = \text{velocity}, T = \text{tension}, \mu = mass/length$)
        \item $v = \text{velocity of sound wave in air}$
          $= 332 m/s$
        \item Intensity of a wave ($J$)\\
          $I = \frac{P}{A}$\\
          $P = \text{power emitted by source} = \frac{\Delta E}{\Delta t} = \text{energy emitted per unit time}$\\
          $A = \text{area} = 4 \pi r^2$\\
          $r = \text{radius(with the source at the center)}$
        \item Intensity level ($\beta$)\\
          $\beta = 10 log_{10} \frac{J}{J_0}$\\
          $I = \text{Intensity at a point from the source of sound}$\\
          $I_0 = 10^{-2} W/m^2 = \text{reference intensity level}$
\end{enumerate}
\end{document}
